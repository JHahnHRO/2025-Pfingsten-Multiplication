% !TeX spellcheck = de_DE
\documentclass[fontsize=11pt,fleqn,a4paper]{scrartcl}

\author{Johannes Hahn}
\title{Schnelle Multiplikation}
\subtitle{$O(n^2)$ ist für Anfänger}
\date{06.06.25 -- 09.06.25}


%%% General math packages

% AMS
\usepackage{amsmath,  
            amssymb,  % Symbols
            amsthm,   % provides theorem environments
            amsfonts  % fonts like \mathbb and \mathfrak
            }

% useful things for math typesetting like \smash, \psmallmatrix, ...
%\usepackage{mathtools}

% For \Set and \set. Automatic resizing of the curly braces and the middle vertical line
%\usepackage{braket}

% For even more extensible arrows
%\usepackage{extpfeil}

% Blockmatrices
%\usepackage{multirow}

% and brakets around matrix rows
%\usepackage{bigdelim}

% for \mathscr
%\usepackage{mathrsfs}
%%% Own symbols and operators
\newcommand{\IN}{\mathbb{N}}
\newcommand{\IZ}{\mathbb{Z}}
\newcommand{\IQ}{\mathbb{Q}}
\newcommand{\IR}{\mathbb{R}}
\newcommand{\IC}{\mathbb{C}}
\newcommand{\IK}{\mathbb{K}}

\DeclarePairedDelimiter{\abs}{\lvert}{\rvert}
\DeclarePairedDelimiter{\norm}{\lVert}{\rVert}
\DeclarePairedDelimiter{\ceil}{\lceil}{\rceil}
\DeclarePairedDelimiter{\floor}{\lfloor}{\rfloor}

\newcommand{\isomorphic}{\cong}

\renewcommand{\Im}{\operatorname{\mathfrak{Im}}}
\renewcommand{\Re}{\operatorname{\mathfrak{Re}}}

\newcommand{\dd}{\,\textrm{d}}

\DeclareMathOperator{\id}{id}
\DeclareMathOperator{\tr}{tr}
\DeclareMathOperator{\casimir}{\mathbf{\Omega}}
\DeclareMathOperator{\Hom}{Hom}
\DeclareMathOperator{\End}{End}
\DeclareMathOperator{\Aut}{Aut}
\DeclareMathOperator{\Sym}{Sym}
\DeclareMathOperator{\STF}{STF}

\DeclareMathOperator{\sgn}{sgn}
\DeclareMathOperator{\ord}{ord}

\DeclareMathOperator{\im}{im}


\newcommand{\vc}[1]{\ensuremath{\mathbf{#1}}}

\newcommand{\mapsfrom}{\mathrel{\reflectbox{\ensuremath{\mapsto}}}}

%%% Theorem environments

% Define name strings
\newcommand{\captionstringtheorem}{Theorem}
\newcommand{\captionstringcentralquestion}{Central Question}
\newcommand{\captionstringmaintheorem}{Main Theorem}
\newcommand{\captionstringlemma}{Lemma}
\newcommand{\captionstringcorollary}{Corollary}
\newcommand{\captionstringlemmadef}{Lemma and definition}
\newcommand{\captionstringtheoremdef}{Theorem and definition}
\newcommand{\captionstringdefinition}{Definition}
\newcommand{\captionstringproposition}{Proposition}
\newcommand{\captionstringexample}{Example}
\newcommand{\captionstringconjecture}{Conjecture}
\newcommand{\captionstringconvention}{Convention}
\newcommand{\captionstringremark}{Remark}


% Definition of two similar styles
\newtheoremstyle{dotless} % Name
			{\bigskipamount}    % Space above
			{\bigskipamount}    % Space below
			{\nopagebreak}      % Body font, also suppress pagebreak between "Theorem 3.14:" and text
			{}                  % Indent amount
			{\bfseries}         % Theorem head font
			{:}                 % Punctuation after theorem head
			{\newline}          % Space after theorem head
			{}                  % Theorem head spec (can be left empty, meaning 'normal')
\newtheoremstyle{dotless2} % Name
			{\bigskipamount}    % Space above
			{0.0em}             % Space below
			{}                  % Body font
			{}                  % Indent amount
			{\bfseries}         % Theorem head font
			{:}                 % Punctuation after theorem head
			{0.5em}             % Space after theorem head
			{}                  % Theorem head spec (can be left empty, meaning 'normal')
% "3.14 Theorem" instead of "Theorem 3.14"
\swapnumbers

\newcounter{theoremnumber}
\numberwithin{theoremnumber}{section}

\theoremstyle{dotless}
\newtheorem{theorem}[theoremnumber]{\captionstringtheorem}
\newtheorem{maintheorem}[theoremnumber]{\captionstringmaintheorem}
\newtheorem{centralquestion}[theoremnumber]{\captionstringcentralquestion}
\newtheorem{theoremdef}[theoremnumber]{\captionstringtheoremdef}
\newtheorem{lemma}[theoremnumber]{\captionstringlemma}
\newtheorem{lemmadef}[theoremnumber]{\captionstringlemmadef}
\newtheorem{corollary}[theoremnumber]{\captionstringcorollary}
\newtheorem{proposition}[theoremnumber]{\captionstringproposition}
\newtheorem{definition}[theoremnumber]{\captionstringdefinition}
\newtheorem{example}[theoremnumber]{\captionstringexample}
\newtheorem{conjecture}[theoremnumber]{\captionstringconjecture}

\newtheorem*{convention}{\captionstringconvention}

\theoremstyle{dotless2}
\newtheorem{remark}[theoremnumber]{}


%%% Math style file

% Sets the strictness of handling page breaks inside align-Umgebungen
% Level [1] breaks will be avoided if possible,
% Levels [2], [3], [4] are increasingly relaxed.
\allowdisplaybreaks[1]


%%%% Everything tikz

\usepackage{tikz}
\usetikzlibrary{arrows}
\usetikzlibrary{cd}
\usetikzlibrary{babel}

%%%% Styles for algorithms

\usepackage{listings}
\lstset{%
	basicstyle = \ttfamily\small,
	tabsize = 3
}
% Use theorem environment for algorithm descriptions
\theoremstyle{dotless}
\newtheorem{algorithm}{Algorithmus}
% Change numbering of algorithms to include chapter
\renewcommand{\thealgorithm}{A\arabic{algorithm}}
%%%% Tables and figures
\numberwithin{table}{section}
\numberwithin{figure}{section}
\usepackage{graphicx}

\usepackage{tabularx,ragged2e}
\newcolumntype{C}{>{\Centering\arraybackslash}X} % centered "X" column
\newcolumntype{L}{>{\Centering\arraybackslash}{\hsize=.5\hsize}X} % centered "X" column
\newcolumntype{S}{>{\Centering\arraybackslash}{\hsize=.5\hsize}X} % centered "X" column



%%% Language preamble for german

% Language itself
\usepackage[ngerman]{babel}
% Font encoding to represent umlauts correct in PDFs documents instead of combining them from other
% characters like "u instead of ü. Do not use with XeLaTeX or LuaTeX
%\usepackage[T1]{fontenc}
% Encoding of the source code. Do not use with XeLaTeX or LuaTeX
%\usepackage[utf8]{inputenc}


% Language specific strings
%   Names of theorem environments

\addto\captionsngerman{
	% \see command from makeidx
%	\@ifpackageloaded{makeidx}{
%		\renewcommand{\seename}{siehe}
%	}{}
	\renewcommand{\captionstringtheorem}{Satz}
	\renewcommand{\captionstringmaintheorem}{Hauptsatz}
	\renewcommand{\captionstringcentralquestion}{Zentrale Fragestellung}	
	\renewcommand{\captionstringlemma}{Lemma}
	\renewcommand{\captionstringcorollary}{Korollar}
	\renewcommand{\captionstringlemmadef}{Lemma und Definition}
	\renewcommand{\captionstringtheoremdef}{Satz und Definition}
	\renewcommand{\captionstringdefinition}{Definition}
	\renewcommand{\captionstringproposition}{Proposition}
	\renewcommand{\captionstringexample}{Beispiel}
	\renewcommand{\captionstringconjecture}{Vermutung}
	\renewcommand{\captionstringconvention}{Vereinbarung}
	\renewcommand{\captionstringremark}{Bemerkung}
}
\makeatletter
%\@ifpackageloaded{biblatex}{
%\DefineBibliographyStrings{german}{
%	bibliography = {Bibliographie},
%	references = {Referenzen}
%}
%}{}
\@ifpackageloaded{exframe}{
	\exerciseconfig{termsheet}{Blatt}
	\exerciseconfig{termsheets}{Blätter}
	\exerciseconfig{termproblem}{Aufgabe}
	\exerciseconfig{termproblems}{Aufgaben}
	\exerciseconfig{termsolution}{Lösung}
	\exerciseconfig{termsolutions}{Lösungen}
	\exerciseconfig{termpoint}{Punkt}
	\exerciseconfig{termpoints}{Punkte}
}
\makeatother

%%% Text styles

% For the interrobang
%\usepackage{textcomp}

% Additional underlining options, especially dotted underlining, line breaks in underlined text etc.
\usepackage[normalem]{ulem} % option not to change look of \emph{}
\newcommand*{\udot}{\dotuline}
\AtBeginEnvironment{definition}{\renewcommand{\emph}{\udot}}
\AtBeginEnvironment{lemmadef}{\renewcommand{\emph}{\udot}}
\AtBeginEnvironment{theoremdef}{\renewcommand{\emph}{\udot}}


% Skip lengths
\setlength{\parindent}{0em}
\setlength{\parskip}{0em}


% Pimp enumerate and itemize environments. More counter options and resuming numbering
\usepackage{enumitem}

% Style of enumerations and itemizations
\renewcommand{\labelenumi}{\alph{enumi}.)}  % Counter enumi wird immer als a.) b.) c.) dargestellt.
\renewcommand{\labelenumii}{\roman{enumii}.)}  % Counter enumii wird immer als i.) ii.) iii.) dargestellt.


\usepackage{csquotes}

%\usepackage{multicol}

% For measurements with unit
\usepackage{siunitx}
\DeclareSIUnit{\bit}{bit}
\DeclareSIUnit{\byte}{B}

% For colorful notes in the margin
% Load after tikz!
%\usepackage[backgroundcolor=red!20,textsize=footnotesize]{todonotes}
%\newcommand{\progressMarker}[1]{\todo[backgroundcolor=green!20,linecolor=green]{#1}}

%%%% Bibliography with BibTeX only

\usepackage[numbers]{natbib}
\bibliographystyle{plain}


%%%% Bibliography with biblatex + bibtex

\usepackage[
	style=alphabetic,
	sorting=nty,
	url=false,
	natbib=true,
	backend=biber,  % Biber backend
	defernumbers=true
]{biblatex}
% The .bib-file(s) for this document
\addbibresource{main.bib}


%%%% Bibliography with biblatex + bibtex

\usepackage[
	style=alphabetic,
	sorting=nty,
	url=false,
	natbib=true,
	backend=biber,  % Biber backend
	defernumbers=true
]{biblatex}
% The .bib-file(s) for this document
\addbibresource{main.bib}



% !! Hyperref before imakeidx !!
%%% PDF stuff
%%%
%%% Include hyperref before imakeidx !!

\usepackage[
    pdfpagelabels=true,
    plainpages=false,
    pdfencoding=unicode
]{hyperref}

\hypersetup{
    colorlinks=true,
    linkcolor=red,
    urlcolor=blue,
    citecolor=blue,
    linktocpage=true, % Page numbers will be the links in t.o.c instead of the headings themselves
    pdfpagelayout={OneColumn},
    pdfstartview= % empty to cause the viewer to use its preferred behaviour instead of dictating a behaviour at opening of the document
}

\makeatletter
{\hypersetup{
    pdfinfo=
        {
        Title={\@title},
        Author={Johannes Hahn, Lena-Elisa Schneegans},
        Keywords={Philosophy of science, scientific method, metaphysics, epistomology, },
        Subject={Philosophy of science}
    }
}}
\makeatother
%%% Index

% for multiple indices
\usepackage{imakeidx}
\indexsetup{
	level=\chapter*,                % Headings of indices will be displayed as \chapter* would
	toclevel=chapter,               % Indices will appear in the table of contents like a chapter
	headers={\indexname}{\indexname}% Header will contain the name
	}
% creates a new index
\makeindex[
	% internal name of the index
	name=terms,
	% Displayed name
	title=Stichwortverzeichnis,
	% Will be listed in the table of contents
	intoc,
	% will be displayed in two columns
	columns=2,
	% will be generated automatically with makeindex and the given program options inlcuding the
	% *.ist file (index style) that is to be used.
	program=makeindex,options=-s terms.ist
	]



\usepackage{exframe}
\exercisesetup{
	solutions=false,     % display solutions?
	solutionhref=true    % hyperlink solutions to problems
}
\exercisestyle{
	solutionbelow=sheet,% Display solutions under each sheet
	problembysheet,      % number problems with sheet, e.g. problem 2.4.
%	pagebysheet,         % Number pages with sheets
	pointsat=start,
	subpointsat=margin,
	fracpoints=true      % fractional points
}

%% Config of sheets
\exerciseconfig{toclevelsheet}{subsection}

\exerciseconfig{insertsheetbefore}{\setcounter{sheet}{\thesection}}
\exerciseconfig{insertsheettitle}{\subsection*{\getexerciseconfig{termproblems}}}
\exerciseconfig{insertsheetafter}{}

%% Config of problems

\exerciseconfig{toclevelproblem}{}
\exerciseconfig{styletitleproblem}{}

\exerciseconfig{skipproblemabove}{\bigskipamount}
\exerciseconfig{skipproblembelow}{\bigskipamount}
\exerciseconfig{skipproblemtitle}{1pt}

\exerciseconfig{composetitlesheet}[2]{\getexerciseconfig{termsheet} #1 \exerciseifnotempty{#2}{#2}}
\exerciseconfig{composetitleproblem}[2]{\getexerciseconfig{termproblem} \getexerciseconfig{composeitemproblem}{#1}\exerciseifnotempty{#2}{ -- #2}}
\showprobleminfo{difficulty}

\exerciseconfig{composepointspair}[2]{%
  \ifdim#2pt=0pt% If zero bonus points
   \getexerciseconfig{composepoints}{#1}%
  \else\ifdim#1pt=0pt% If only bonus points
   +\getexerciseconfig{composepointsnum}{#2}~Bonus%
  \else% If both non-zero
   \getexerciseconfig{composepoints}{#1}+%
   \getexerciseconfig{composepointsnum}{#2}~Bonus%
  \fi\fi%
}
\exerciseconfig{composepointspairstart}[2]{\hfill(%
   \getexerciseconfig{composepointspair}{#1}{#2}%
)}
\exerciseconfig{composepointspairmargin}[2]{(%
  \ifdim#2pt=0pt% If zero bonus points
   \getexerciseconfig{composepointsnum}{#1}%
  \else\ifdim#1pt=0pt% If only bonus points
   +\getexerciseconfig{composepointsnum}{#2}~B%
  \else% If both non-zero
   \getexerciseconfig{composepointsnum}{#1}+%
   \getexerciseconfig{composepointsnum}{#2}~B%
  \fi\fi%
)}

%% Config of subproblems
\exerciseconfig{countersubproblem}{\alph{subproblem}.)}

%% Config of solutions
\exerciseconfig{styletitlesolutions}{\Large}
\exerciseconfig{insertsolutionsbefore}{\pagebreak}



\begin{document}

\maketitle

\tableofcontents
\pagebreak

\setcounter{section}{-1}


\section{Crashkurs theoretische Informatik}\label{theory}
% !TeX root = main.tex
% !TeX spellcheck = de_DE

\subsection{Landau-Notation}\label{landau-notation}

\begin{remark}
    Wir wollen uns damit auseinandersetzen, welche verschiedenen Algorithmen es gibt, um große Zahlen miteinander zu multiplizieren, und wie sie sich voneinander unterscheiden. Wir wollen insbesondere die Geschwindigkeit dieser Algorithmen untersuchen.

    \medskip
    Das ist nur leider ein relativ schwieriges Problem. Offensichtlich ist die Laufzeit vom Input abhängig. Zehn-stellige Zahlen und 1.000.000-stellige Zahlen zu multiplizieren wird offensichtlich unterschiedlich lange dauern. Also wollen wir nicht zwei Zahlen miteinander vergleichen und herausfinden, welche die größere ist, sondern zwei Funktionen, die uns die Laufzeit eines Algorithmus in Abhängigkeit von der Größe der Inputs berechnet.

    \medskip
    Es ist nicht von vornherein offensichtlich, wie das geschehen soll. Als Beispiel betrachte man
    \[f(n) = 100n+2 \quad\text{vs.}\quad g(n) = 3n^2\]
    Welches ist die \enquote{bessere} dieser Funktionen? Für $n\leq 33$ ist $g(n)$ die kleinere Zahl, für größere $n$ hingegen $f(n)$. Es scheint als sei keiner der beiden hypothetischen Algorithmen, die hinter diesen beiden Funktionen stehen, optimal für alle Inputs. Stattdessen würde man viel mehr einen hybriden Algorithmus benutzen wollen, der das eine Verfahren für kleine und das andere für große Inputs implementiert. Aber natürlich weiß man das erst hinterher, wenn man die anderen Verfahren schon kennt\ldots
\end{remark}

\begin{remark}
    Zusätzlich kommt hinzu, dass keine der Konstanten 100,2,3 in den Funktionen zuverlässig vorhergesagt werden kann. Wie viele Sekunden ein konkretes Programm für einen bestimmten Input tatsächlich benötigt, hängt von einer Vielzahl von Faktoren ab, die völlig unabhängig vom verwendeten Algorithmus sind:
    \begin{enumerate}
        \item Auf welchem Rechner wird der Algorithmus ausgeführt? Hat er einen \SI{2}{\giga\hertz} oder \SI{4}{\giga\hertz} Prozessor? Wie groß sind die L1, L2, L3-Caches? Ist es ein Laptop? Ist er gerade im Batteriemodus? Scheint gerade die Sonne auf den Bildschirm?
        \item In welcher Programmiersprache ist das Programm geschrieben worden und wie wurde es kompiliert? C ? C++? Java? C++11 oder C++23 ? Welcher Compiler auf welcher Optimierungsstufe wurde verwendet? Java 17, 21 oder 24 ? Ist eine frische JVM oder eine aufgewärmte verwendet worden?
        \item Welche anderen Prozesse liefen gerade auf dem Rechner?
        \item \ldots
    \end{enumerate}
    Und selbst wenn wir in der Lage wären, einige dieser Faktoren gut genug zu verstehen, um eine belastbare Vorhersage der Laufzeit zu machen, würde solch eine Vorhersage nicht lange halten. Im Herbst dieses Jahres\footnote{Das wäre das Jahr 2025 für alle, die dies in der Zukunft lesen. Seid gegrüßt Zeitreisende!} wird Java 25 veröffentlicht werden; möglicherweise ist unser Algorithmus auf der neuen JVM ein bisschen schneller. Unsere Analyse zum jetzigen Zeitpunkt wird jedenfalls nicht berücksichtigen können, was in den nächsten Monaten an Mikro-Optimierungen implementiert werden wird.
\end{remark}

\begin{remark}
    Auf der anderen Seite ist die exakte Anzahl an Millisekunden vermutlich gar nicht so relevant und uns interessiert viel mehr, ob wir überhaupt warten müssen, ob es sich lohnt einen frischen Kaffee zu holen (oder eine Tafel Schokolade zu essen), ob wir das Ergebnis überhaupt in diesem Leben noch erfahren werden. Dafür ist der genaue Wert dieser Konstanten vermutlich nicht relevant.

    \medskip
    Wir wollen also eine eher grobe Analyse durchführen, die von möglichst vielen hässlichen Faktoren der Realität unabhängig ist.

    Wir wollen insbesondere keine exakten Zeiten messen. Typischerweise werden wir etwa die Laufzeit unserer Algorithmen nicht in Sekunden, sondern der Anzahl von \enquote{elementaren Operationen} zählen. Was genau \enquote{elementar} ist, hängt vom Kontext ab.

    \smallskip
    Klassisches Thema einer \enquote{Theoretische Informatik I}-Vorlesung wären z.B.\ die verschiedenen Sortieralgorithmen, für die man üblicherweise zählt, wie oft man zwei Elemente der zu sortierenden Liste miteinander vergleichen und ggf. ihre Positionen vertauschen muss, bis man die Liste sortiert hat.

    \smallskip
    Wir werden davon ausgehen, dass wir mit 1- und 2-stelligen Zahlen bereits addieren und multiplizieren können, mit 1.000.000-stelligen Zahlen aber nicht. Wir zählen also die Operationen mit kleinen Zahlen, die wir benötigen, um mit großen Zahlen zu rechnen.

    \smallskip
    Ein anderer Algorithmus, der ein anderes Ziel hat und sich nicht für die Details der Multiplikation interessiert, wird vielleicht die Multiplikation beliebiger Zahlen als eine Operation zählen.
\end{remark}

\begin{remark}
    Weil wir insbesondere am Multiplikationsproblem \emph{sehr großer} Zahlen interessiert sind, werden wir außerdem in vielen Fällen vernachlässigen, wenn ein Algorithmus für kleine Inputs suboptimal ist. Wie bereits angemerkt, kann man das beheben, indem man verschiedene Algorithmen miteinander geschickt kombiniert. Für die theoretische Analyse soll uns aber vor allem das Verhalten für $n\to\infty$ interessieren.\footnote{Man beachte, dass wir damit wirklich die Grenzen der Realität verlassen. Das beobachtbare Universum ist verdammt groß\cite{adams1979hitchhikersguide}, aber trotzdem endlich. Es gibt eine Größe von Inputs, die einfach nicht ins Universum passt. Spätestens ab dieser Größe ist es für alle Zwecke der realen Welt sinnlos, sich noch über die Laufzeit von Algorithmen zu unterhalten. Wir werden es trotzdem tun, einfach weil's Spaß macht!}
\end{remark}

\begin{definition}[Landau-$O$-Notation]
    Es sei $f:\IN\to\IR_{\geq 0}$ eine Funktion. Die Menge $O(f)$ definieren wir als
    \[O(f) := \Set{g:\IN\to\IR_{\geq 0} | \exists c,n_0>0 \forall n\geq n_0: f(n) \leq c g(n)}\]
    Wir schreiben der Kürze halber $O(f(n))$, wenn wir $O(f)$ meinen, also z.B.\ $O(n^2)$ anstatt $O(n\mapsto n^2)$.
\end{definition}

\begin{remark}
    $g\in O(f)$ bedeutet anschaulich, dass $g(n)$ höchstens so schnell wächst wie $f(n)$ für sehr große Inputs.

    $f(n)=n^2$ hat z.B.\ die Eigenschaft, dass sich $f(n)$ vervierfacht, wenn $n$ sich verdoppelt. $g\in O(n^2)$ bedeutet somit, dass $g(2n)$ nicht viel mehr als viermal größer als $g(n)$ ist, zumindest wenn $n$ hinreichend groß ist.
\end{remark}
\begin{remark}
    In dem meisten Fällen kann man die Definition auch umformulieren in
    \[g\in O(f) \iff \limsup_{n\to\infty} \frac{g(n)}{f(n)} < \infty\]
    da man meistens nicht $f(n)=0$ für unendlich viele $n$ hat. Man betrachtet i.d.R.\ monoton wachsende Funktionen $f$ mit $\lim_{n\to\infty} f(n) = \infty$ für die Analyse.
\end{remark}
\begin{remark}
    Man beachte aber, dass dies keine Aussage über die exakten Werte von $g(n)$ macht. $0,0000001 n^2$ ist genauso in $O(n^2)$ wie $1.000.000n^2 + 10^{10^{10^{42}}}$. Wenn ein Algorithmus einfach genug aufgebaut ist, kann man manchmal auch exakte Werte ermitteln, aber es ist typischerweise einfacher, nur auf die $O$-Komplexität zu schauen, weil man dann solche Unterschiede ignorieren kann.

    Es kommt auch häufig vor, dass man den Term, der die Komplexität bestimmt, gut versteht, aber kleinere Terme ignorieren will. Dann würde man z.B.\ herausfinden, dass die betrachtete Funktion die Form $3n^2+O(n)$ hat o.Ä.; der führende Term $3n^2$ bestimmt die Komplexität -- es ist $O(n^2)$ --, aber die echten Werte sind nicht exakt $3n^2$, die Differenz liegt aber in der kleineren Klasse $O(n)$.
\end{remark}
\begin{remark}
    Man beachte auch, dass die $O$-Notation nur eine Abschätzung nach oben liefert. Wenn wir uns nicht clever anstellen, kann es also passieren, dass wir massiv überschätzen. Es ist z.B.\ völlig korrekt zu sagen, dass $f(n)=n^2+17$ in $O(n^{100})$ enthalten ist.

    \medskip
    Es gibt ähnlich gelagerte Definitionen wie $o(f)$ und $\Theta(f)$, die für untere bzw. genaue Abschätzung gemacht sind.

    \medskip
    Je komplexer ein Algorithmus ist, desto schwieriger ist es, genau zu erkennen, dass man die bestmögliche Abschätzung gefunden hat und in der Tat gibt es Algorithmen, für die die präzise Komplexität nicht bekannt ist.

    \medskip
    Noch schlimmer wird es, wenn man nicht die Komplexität eines einzelnen Algorithmus betrachtet, sondern einfach nur ein bestimmtes zu lösendes Problem vorgibt und sich \emph{alle} Algorithmen anschaut, die dieses Problem lösen könnten. Wenn man einen einzelnen Algorithmus findet, bekommt man eine Abschätzung nach oben, aber um die genaue Komplexität des Problems zu kennen, muss man \emph{alle} Algorithmen betrachten und beweisen, dass keiner von ihnen in eine strikt bessere Laufzeitklasse fällt. Damit sind nicht alle \emph{bekannten} Algorithmen gemeint, sondern alle nur denkbaren!

    \medskip
    Selbst für viele elementare Probleme ist die optimale Laufzeit nicht bekannt! Für Ganzzahl-Multiplikation \emph{vermutet} man, dass die Komplexität $\Theta(n\log(n))$ ist, aber wir können uns nicht sicher sein. Immerhin wissen wir sei 2019, dass es tatsächlich einen Algorithmus in $O(n\log(n))$ gibt, die obere Grenze also schon einmal stimmt.

    Für Matrix-Multiplikation vermutet man $n^{2+o(1)}$, aber man kennt nicht einmal einen Algorithmus, der dem nahe käme. Der beste bekannte Algorithmus ist in $O(n^{2,371552})$, aber man weiß nicht, wie viel dichter an 2 man kommen kann.
\end{remark}


\begin{example}
    \begin{enumerate}
        \item Es ist immer $f\in O(f)$.
        \item $O(1)$ ist die Menge aller beschränkten Funktionen, also der Funktionen $f$, für ein $c$ existiert mit $f(n)\leq c$ für alle $n\in\IN$.
        \item $O(1) \subsetneq O(n^2) \subsetneq O(n^3) \subsetneq O(n^4) \subsetneq \ldots$
        \item Für Polynome ist nur der Grad entscheidend: Ist $f(n) = a_k n^k+a_{k-1} n^{k-1} + \ldots $, dann $f\in O(n^k)$.
        \item Allgemeiner: $O$-Notation ignoriert \enquote{Terme kleinerer Ordnung}, d.h.\ wenn $g\in O(f)$ ist, dann ist $f+g$ immer noch in $O(f)$.
    \end{enumerate}
\end{example}

\begin{remark}
    Wir verwenden die $O$-Notation nicht nur für die Funktionen, die die Laufzeit von Algorithmen messen, sondern beispielsweise auch für Funktionen, die den Speicherbedarf messen.

    Wir würden also z.B.\ sagen, dass ein Algorithmus bei einem Input der Größe $n$ eine \enquote{Laufzeit von $O(n^6)$ und Speicherbedarf von $O(n^3 \log(n)^2)$ hat}. Damit meinen wir, dass die Funktion, die die Laufzeit in Abhängigkeit von $n$ misst, in $O(n^6)$ enthalten ist, und die Funktion, die den Speicherbedarf in Abhängigkeit von $n$ misst, in $O(n^3\log(n)^2)$.
\end{remark}
% !TeX root = main.tex
% !TeX spellcheck = de_DE

\begin{sheet}
    \begin{problem}
        Betrachte den folgenden Algorithmus für Multiplikation von $l\times m$- und $m\times n$-Matrizen.
        \begin{lstlisting}
Matrix multiply(Matrix a, Matrix b){
    int l = nrOfRows(a);
    int m = nrOfColumns(a);
    if(m != nrOfRows(b)){
        throw new ArithmeticException("Matrix multiplication undefined");
    }
    int n = nrOfColumns(b);

    var C = new Matrix(l,n);

    for(int i=0; i<l; i++){
        for(int j=0; j<n; j++){
            for(int k=0; k<m; k++){
                C[i,j] += A[i,k] * B[k,j];
            }
        }
    }
    return C;
}
        \end{lstlisting}

        Welche Laufzeit- und Speicher-Komplexität hat dieser Algorithmus um zwei $n\times n$-Matrizen miteinander zu multiplizieren?
    \end{problem}

    \begin{problem}[title={Ein paar übliche Komplexitätsklassen}]
        \begin{subproblem}
            Es seien $r,s$ reelle Zahlen. Zeige
            \[r<s \implies O(n^r) \subsetneq O(n^s)\]
            Zeige insbesondere, dass dies eine echte Inklusion ist, d.h.\ dass es eine Funktion $g\in O(n^s) \setminus O(n^r)$ gibt.
        \end{subproblem}
        \begin{subproblem}
            Zeige $\log(n),\log(\log(n)), \log(\log(\log(n))), \ldots \in O(n^\varepsilon)$ für jedes $\varepsilon>0$.
        \end{subproblem}
        \begin{subproblem}
            Zeige $n^r \in O(c^n)$ für jedes $r>0$ und $c>1$.
        \end{subproblem}
    \end{problem}

    \begin{problem}
        Es seien $f,g,F,G:\IN\to\IR_{\geq 0}$ Funktionen mit $f\in O(F), g\in O(G)$.
        \begin{subproblem}
            Zeige, dass $f+g\in O(F+G)$.
        \end{subproblem}
        \begin{subproblem}
            Zeige, dass $f\cdot g\in O(F\cdot G)$.
        \end{subproblem}
        \begin{subproblem}
            Folgere, dass $\log(n)^r, \log(\log(n))^r, \log(\log(\log(n)))^r, \ldots$ alle in $O(n^\varepsilon)$ sind für jedes $\varepsilon>0$.

            Folgere allgemeiner, dass Funktionen der Form $n^s\cdot \log(n)^{r_1} \cdot \log(\log(n))^{r_2} \cdot \log(\log\log(n))^{r_2}\cdots$ alle in $O(n^{s+\varepsilon})$ sind für jedes $\varepsilon>0$.
        \end{subproblem}
    \end{problem}

\end{sheet}

\pagebreak


\section{Level 0: \texorpdfstring{$O(n^2)$}{O(n²)} wie in der Grundschule}\label{slow_multiplication}
% !TeX root = main.tex
% !TeX spellcheck = de_DE

\subsection{Das Problem}

\begin{remark}
    Wir wollen effizient multiplizieren. Was \enquote{effizient} heißt, hängt von der Problemstellung ab. Konkret ist unser Problem folgendes:

    \medskip
    Gegeben zwei natürliche Zahlen $A,B$, dargestellt durch ihre $n$ Ziffern in Basis $g$
    \[A=\sum_{i=0}^{n-1} a_i g^i,\quad B=\sum_{i=0}^{n-1} b_i g^i\]
    mit $0\leq a_i,b_i<g$, berechne die Ziffern $c_0, c_1, \ldots, c_{2n-1}$ des Produkts $C:=A\cdot B$.

    \medskip
    Wir messen die Laufzeit von Multiplikationsalgorithmen in Abhängigkeit von $n$. Wir nehmen an, dass Operationen auf einzelnen Ziffern eine konstante Laufzeit haben.
\end{remark}

\begin{remark}
    Die Annahme, dass Operationen mit einzelnen Ziffern eine konstante Laufzeit haben, ist in der Theorie dadurch gerechtfertigt, dass man eine Additions- und Multiplikationstabelle vorberechnen kann. Dann hat jede Addition oder Multiplikation von zwei Ziffern die Laufzeit von \enquote{Schau einen Eintrag in eine $n\times n$ Tabelle nach}.

    So funktionierte das letztendlich auch in der Grundschule. Das Einmaleins haben wir auswendig gelernt und schriftliches Addieren und Multiplizieren von größeren Zahlen nutzt den \enquote{Lookup} im Gedächtnis.
\end{remark}

\begin{remark}
    In Computerhardware sind i.A.\ keine Lookup-Tabellen implementiert. Die offensichtliche Wahl der Basis wäre $g=2$. Die Addition und Multiplikation von einzelnen Bits sind einfach\footnote{Wie \enquote{einfach} es auch sein mag, wenige hundert Atome große Schaltkreise zu bauen...} direkt in Hardware zu implementieren, sodass kein Lookup nötig ist.

    \smallskip
    Etwas weniger offensichtlich ist es, auf \enquote{Wörtern} statt Bits zu arbeiten, d.h. $g=2^8$, $2^{16}$ (richtig alte Hardware), $2^{32}$ (alte Hardware und einige moderne Mobilgeräte) oder $2^{64}$ (moderne Hardware, insb. PCs) zu wählen. Manche Hardware unterstützt sogar noch höhere Wortgrößen.

    \smallskip
    Während Speicher für eine $256\times 256$ große Tabelle mit je $\SI{2}{\byte}$ großen Einträgen sicher noch verkraftbar wäre, ist eine $2^{32}\times 2^{32}$ Tabelle schon zu groß und eine $2^{64}\times 2^{64}$  Lookup-Tabelle völlig unrealistisch: $2^{64}$ viele $\SI{64}{\bit}$ große Einträge sind $\SI{1.476e8}{\tera\byte}$ und $2^{128}$ viele $\SI{128}{\bit}$ große Einträge sind $\approx \SI{5.445e27}{\tera\byte}$. Kein Datacenter der Welt hat diese Kapazität! Das ist ein Vielfaches der Gesamtgröße des Internets.

    \smallskip
    Also wird auch in diesen Fällen das Ergebnis einer einzelnen Multiplikation mit einer speziellen Instruktion direkt von der Hardware berechnet. Die Schaltkreise sind allerdings schon deutlich komplexer. Und in der Tat sind die Schaltkreise für größere Wörter i.d.R.\ zusammengesetzt aus Schaltkreisen, die Addition und Multiplikation kleinerer Wörter implementieren, d.h.\ es wird ein passender Multiplikationsalgorithmus in Hardware implementiert.

    \smallskip
    Solange alle Multiplikation von zwei Wörtern stets durch den gleichen Schaltkreis ermittelt wird, kann aber ebenfalls davon ausgegangen werden, dass die Laufzeitkosten konstant sind.
\end{remark}

\begin{remark}
    Es ist jedoch nicht unbedingt klar, dass die beiden Schaltkreise für Addition und Multiplikation gleich schnell operieren. In der Tat war auf echter Hardware lange ein deutlicher Unterschied zwischen den Laufzeiten einer Additions- und einer Multiplikationsinstruktion vorhanden. Teilweise war Multiplikation eine Größenordnung langsamer als Addition. Aus diesem Grund hat es sich auch in der theoretischen Analyse eingebürgert, nicht alle Operationen eines Multiplikationsalgorithmus zu zählen, sondern nur die Multiplikationen, weil diese mehr ins Gewicht fallen.
    
    \smallskip
    Wir werden dieser Tradition auch folgen, aber trotzdem Buchführen, wie viele Additionen ein Algorithmus jeweils benötigt, wenn auch im Zweifelsfall etwas weniger genau. Insbesondere werden wir keine Additionen zählen, die Schleifenvariablen inkrementieren. Die sind Teil des Konstrukts \enquote{Schleife}, nicht unseres spezifischen Algorithmus. (Und wie die Schleifen tatsächlich ausgeführt werden, ist außerdem zu schwer vorhersagbar, um konkrete Anzahlen an Additionen vorhersagen zu können)
\end{remark}

\begin{remark}
    Der offensichtliche Algorithmus für Multiplikation beliebiger natürlicher Zahlen ergibt sich direkt durch Anwendung des Distributivgesetzes:
    \[\left(\sum_{i=0}^{n-1}\ a_i g^i \right)\cdot\left(\sum_{j=0}^{n-1} b_j g^j\right) = \sum_{i,j} (a_i b_j) g^{i+j} = \sum_{k=0}^{2n-2} \left(\sum_{i+j=k}a_i b_j\right)g^k\]
    Indem wir alle $n^2$ Produkte $a_i b_j$ berechnen und geeignet aufsummieren, erhalten wir das Produkt $A\cdot B$.

    \medskip
    In der Grundschule lernt man meist ein tabellarisches Verfahren, das schematisch in etwa so abläuft, dass in jeder Zeile das Produkt einer einzelnen Ziffer $a_i$ von $A$ mit $B$ ausgerechnet wird, um $i$ viele Dezimalstellen verschoben wird (was die Multiplikation mit $10^i$ darstellt), sowie am Ende alle Zeilen aufsummiert werden.

    \smallskip
    In jeder Zeile werden dafür die Produkte $a_i\cdot b_j$ ausgerechnet. Grob vereinfacht würde das also so aussehen:

    \[\begin{array}{c c c c c c c c}
        a_2 & a_1 & a_0     & \cdot   & b_3     & b_2     & b_1     & b_0     \\
        \hline
        &     &         &         & a_0 b_3 & a_0 b_2 & a_0 b_1 & a_0 b_0 \\
        &     &         & a_1 b_3 & a_1 b_2 & a_1 b_1 & a_1 b_0 & 0       \\
        &     & a_2 b_3 & a_2 b_2 & a_2 b_1 & a_2 b_0 & 0       & 0       \\
        \hline
        & c_6 & c_5     & c_4     & c_3     & c_2     & c_1     & c_0
    \end{array}\]

    Natürlich sind die Produkte $a_i\cdot b_j$ der einzelnen Ziffern i.A.\ größer als eine Ziffer, sodass man in jeder Zeile und von einer Zeile zur nächsten jeweils die Überträge beachten muss.
\end{remark}

\begin{algorithm}[Schriftliches Multiplizieren]
    \label{alg:slow_multiplication}
    Wir benutzen die Schulmethode direkt. Einziger Unterschied ist, dass wir ein laufendes Zwischenergebnis pflegen und dies kontinuierlich updaten, anstatt $n$ viele Zwischenergebnisse zu speichern und erst am Ende alle aufzuaddieren.

    %! suppress = MissingLabel
    \begin{lstlisting}
    Array<Digit> multiply(Array<Digit> a, Array<Digit> b){
      // initialize with a.length + b.length zeros
      Array<Digit> c := {0,0,...,0};

      for(i := 0; i<a.length; i++){
        Digit z := 0;

        for(j := 0; j<b.length; j++){
          // p := c[i+j] + a_i * b_j + z is at most a 2-digit number !!
          Digits {p_1,p_0} := c[i+j] + a[i] * b[j] + z;

          c[i+j]    := p_0;
          z         := p_1;
        }
        c[i + b.length] := z;
      }

      return c;
    }
    \end{lstlisting}
\end{algorithm}

\begin{proposition}[Korrektheit]
    Der Algorithmus~\ref{alg:slow_multiplication} ist korrekt. Er benötigt $n^2$ Multiplikation, $O(n^2)$ Additionen und benötigt $n+O(1)$ Speicher.
\end{proposition}
\begin{proof}
    Das meiste ist relativ klar. Wir beginnen, indem wir $C$ mit den Ziffern $000\ldots 000$ initialisieren. Dann addieren wir für jedes $i$ die Zahl $a_i g^i\cdot B$ zu $C$, indem wir $a_i \cdot B$ berechnen und auf die Ziffern $\ldots c_{i+2}c_{i+1}c_i$ addieren, und speichern das Ergebnis wieder in $C$.

    \smallskip
    Wir addieren in jeder Iteration der inneren Schleife die 1er Ziffer des Produkts $p:=a_i \cdot b_j$ an die korrekte Stelle $c_{i+j}$. Wir addieren außerdem den Übertrag der letzten Iteration an diese Stelle und merken uns den Übertrag, den wir dabei erhalten, für die nächste Iteration.

    \smallskip
    Damit das funktioniert, müssen wir die mit \lstinline{!!} markierte Behauptung beweisen:

    Das Produkt zweier Ziffern $a_i,b_j\leq g-1$ ist höchstens $(g-1)^2=g^2-2g+1$ und die Addition von zwei weiteren Ziffern $c,z\leq g-1$ führt zu einem Maximum von $g^2-1$, was echt kleiner $g^2$ ist, also wirklich maximal zwei Ziffern hat. Der Übertrag $z$ ist also auch in der nächsten Iteration maximal eine Ziffer groß.
\end{proof}

\begin{remark}
    Wie wir eben gezeigt haben, ist es möglich, einen Schaltkreis zu bauen, der vier Wörter $a,b,c,z$ als Input nimmt, und in einer einzigen Operation das 2-Wort breite Ergebnis $ab+c+z$ berechnet.
    
    \smallskip
    Es ist von der konkreten Prozessorarchitektur abhängig, ob es tatsächlich eine solche Instruktion gibt. Wenn es sie gibt, sind überhaupt keine Additionsinstruktionen zusätzlich zu den Multiplikationen für unseren Algorithmus nötig.
    
    \smallskip
    Wenn es diesen Schaltkreis nicht gibt, dann gibt es oft trotzdem $(a,b,c)\mapsto ab+c$, womit wir nur noch eine einzige Addition zusätzlich zur Multiplikation benötigen.
\end{remark}

\begin{remark}
    Der Algorithmus ist offenbar nicht davon abhängig, dass $A$ und $B$ die gleiche Anzahl an Ziffern haben. Etwas präziser könnten wir die Laufzeiten also als $O(nm)$ ausdrücken, wobei $n$ und $m$ die Längen der beiden Zahlen sind.
    
    \smallskip
    Darin enthalten ist die Teilaussage, dass die Multiplikation mit einer kleinen Konstanten in $O(n)$ realisierbar ist. Das merken wir uns für später, weil wir es verwenden werden.
\end{remark}

\begin{remark}
    Man beachte auch, dass die asymptotische Laufzeit nicht von der Wahl von $g$ abhängig ist. Wenn wir eine andere Basis $g'$ wählen, dann sind $A$ und $B$ Zahlen mit $\approx \frac{\log(g)}{\log(g')}n$ Ziffern. Die Basis zu ändern, ändert also die Laufzeit nur um einen konstanten Faktor.
\end{remark}


\begin{remark}
    Es ist nicht einmal notwendig, die beiden Inputs $A$ und $B$ bzgl. derselben Basis auszudrücken. In der Tat ist das genau das Prinzip hinter der \enquote{ägyptischen Multiplikation}, auch \enquote{(russische) Bauernmultiplikation} genannt:

    Drückt man eine der beiden Zahlen in Basis $2$ aus, etwa $A$, so reduziert sich das Problem darauf, aus $B$ die Zahlen $2B$, $4B$, $8B$, \ldots zu berechnen, da die Multiplikationen $a_i\cdot B$ entweder $0$ oder $B$ ist (denn $a_i$ ist ja $0$ oder $1$ in Basis $2$), also keinen zusätzlichen Rechenaufwand verursachen:

    \[(a_{n-1} 2^{n-1} + \ldots + a_2 2^2 + a_1 2 + a_0) \cdot B = \sum_{\substack{0\leq i\leq n-1 \\ a_i = 1}} 2^i B\]

    Man kann also mit $B$ beginnen und die Zahl immer wieder zu sich selbst addieren, um $2B$, $4B$, \ldots zu erhalten. Und während dessen, jedes Mal, wenn das passende Bit in $A$ gesetzt ist, eine weitere Addition auf das Zwischenergebnis durchführen.
\end{remark}

\begin{algorithm}[Double-and-add]
    \label{alg:double-and-add}
    Input: $A$ in Binärdarstellung $a_{n-1}2^{n-1}+\ldots a_{2}2^2 +a_1 2 + a_0$, $B$ in beliebiger Basis $b_{m-1} g^{m-1} + \ldots b_2 g^2 + b_1 g+b_0$.

    Output: Das Produkt $C=A\cdot B$ in Basis $g$.

    Pseudocode:
    %! suppress = MissingLabel
    \begin{lstlisting}
    Array<Digit> multiply(Array<Bit> a, Array<Digit> b){
      // calculate the number of base-g-digits in A
      n := floor( a.length * log(2)/log(g) ) + 1;
      c := new Array<Digit>[n + b.length];

      // stores 2^i*b
      b2 := b;
      for(i := 0; i<a.length; i++){
        if(a[i] == 1){
          c := add(c, b2);
        }
        b2 := add(b2,b2);
      }

      return c;
    }
    \end{lstlisting}
\end{algorithm}

\begin{proposition}[Korrektheit und Laufzeit]\label{alg:double-and-add:correctness}
    Der Algorithmus ist korrekt, hat Laufzeit $O(n^2)$ (alles Additionen) und Speicherbedarf $n+O(1)$.
\end{proposition}
\begin{proof}
Übung.
\end{proof}

\begin{remark}
    Da man in der Praxis eben nicht mit beliebigen $g$, sondern mit Zweierpotenzen $g=2^k$ arbeitet, ist für die Berechnung von $\frac{\log(2)}{\log(g)}$ keine Fließkomma-Arithmetik nötig, da sich alles zu $\frac{1}{k}$ kürzt. Die Länge von $A$ in bits ist typischerweise klein genug, damit die Division \texttt{a.length / k} in einer einzigen Hardware-Instruktion berechnet werden kann.

    \smallskip
    Mehr noch: Wenn $g$ selbst eine Zweierpotenz ist, dann besteht eigentlich kein Unterschied zwischen \textt{Array<Bit>} und \texttt{Array<Digit>} und es ist überhaupt keine zusätzliche Aktion nötig.
\end{remark}
% !TeX root = main.tex
% !TeX spellcheck = de_DE

\begin{sheet}
    \begin{problem}
        Wenn $A = \sum_{k=0}^{n-1}a_i g^i$ eine natürliche Zahl mit $\leq n$ Ziffern in Basis $g$ ist, dann ist $\log_g(A)<n$. Wenn $A$ genau $n$ Ziffern hat, also $a_{n-1}>0$ gilt, dann ist $n-1\leq\log_g(A)<n$.
    \end{problem}

    \begin{problem}[title={Parallelisierung}]
        Beschleunige Algorithmus~\ref{alg:slow_multiplication} durch Parallelisieren einiger Schritte um einen konstanten Faktor, sodass der Speicherverbrauch weiterhin $O(n)$ bleibt.
    \end{problem}

    \begin{problem}[title={Square-and-multiply}]
        Es sei ein Monoid $(M,\cdot)$ gegeben und eine Black-box Funktion, die die Multiplikation berechnet.

        \begin{enumerate}
            \item Finde einen Algorithmus, der für gegebenes $x\in M$ und $N\in\IN$ die Potenz $x^N$ in $O(\log_2(N))$ Multiplikationen berechnet.
            \item Was hat das mit dem Problem des Multiplizierens zu tun?
        \end{enumerate}
    \end{problem}

    \begin{problem}
        Beweise Proposition~\ref{alg:double-and-add:correctness}.
    \end{problem}

\end{sheet}



\section{Level 1: \texorpdfstring{$O(n^{1.585})$}{O(n\^{}1.585)} Karatsubas Trick}\label{karatsuba}
% !TeX root = main.tex
% !TeX spellcheck = de_DE

\begin{remark}
    Über Jahrtausende ist keine Methode bekannt gewesen, wesentlich schneller als $O(n^2)$ zu multiplizieren. In der Tat wurde 1956 von Kolmogorov\footnote{Andrey Nikolaevich Kolmogorov, 1903--1987, russ. Mathematiker} vermutet, dass es tatsächlich nicht schneller geht. Jedoch wurde kurz danach von Karatsuba\footnote{Anatoly Karatsuba, 1937--2008, russ. Mathematiker} eine schnellere Methode entdeckt.\footnote{Das Resultat wurde 1960 veröffentlicht. Es wird die Anekdote erzählt, dass Karatsuba das Resultat bereits wenige Tage, nachdem Kolmogorov die Vermutung auf einer Konferenz äußerte, hatte und es noch auf der gleichen Konferenz Kolmogorov mitteilte.}

\medskip
Karatsubas Methode arbeitet rekursiv wie viele anderen, die wir uns ansehen werden: Anstatt in der eigentlich beabsichtigten kleinen Zahlenbasis $g$ (z.B. $g=2$ oder $g=10$) zu arbeiten, werden die Zahlen zunächst bzgl. einer größeren Basis $G$ dargestellt. Um nun zwei Zahlen in Basis $G$ zu multiplizieren, sind mehrere Multiplikationen von Zahlen $<G$ zu erwarten. Diese werden rekursiv mit dem gleichen Algorithmus, aber bzgl. einer Basis $g\leq G'<G$ berechnet. Oft werden die anderen Basen $G,G', \ldots$ als Potenzen von $g$ gewählt, sodass keine zusätzliche Rechenarbeit notwendig ist, um zwischen $G$-Ziffern und $g$-Ziffern umzurechnen. Beispiel: Die Ziffern zur Basis $G=1000$ der Zahl $A=123456789$ sind $a_2=123, a_1=456, a_0=789$.

\medskip
Speziell für Karatsubas Methode wählt man $G$ groß genug, dass die beiden Inputs $A$ und $B$ nur noch aus zwei Ziffern bestehen. Für Zahlen mit zwei Ziffern $A=a_1G+a_0, B=b_1G+b_0$ gilt:
\[AB = \underbrace{a_1 b_2}_{=c_2} G^2 + (\underbrace{a_1 b_0+a_0 b_1}_{=c_1})G+\underbrace{a_0 b_0}_{=c_0}\]
In dieser Form aufgeschrieben, sehen wir genau die $2^2$ Multiplikationen, die der klassische Algorithmus benötigt. Karatsuba sah, dass der mittlere Term auch gleich
\[c_1 = (a_1+a_0)(b_1+b_0)-a_1 b_1-a_0b_0 = (a_1+b_1)(b_1+b_0)-c_2-c_0\]
ist. Somit kann man $c_1$ anstatt mit zwei Multiplikationen und einer Addition auch durch eine Multiplikation und zwei Subtraktionen berechnen. Additionen und Subtraktionen haben typischerweise annähernd die gleiche Laufzeit.
\end{remark}

\begin{algorithm}[Karatsubas Algorithmus]
    \label{alg:karatsuba}
    %! suppress = MissingLabel
    \begin{lstlisting}
    List<Digit> multiply(List<Digit> a, List<Digit> b) {
        int n := max(a.length, b.length);
        if(n <= CUTOFF){
            return slow_multiply(a,b); // algorithm A1
        }
    
        int k := ceil(n/2); // round up if n is odd
        
        low   := multiply(a[0..k-1], b[0..k-1]);
        high  := multiply(a[k..n-1], b[k..n-1]);
        
        mixed_a := add(a[0..k-1], a[k..n-1]);
        mixed_b := add(b[0..k-1], b[k..n-1]);
        
        product := multiply(mixed_a, mixed_b);
        middle := subtract(subtract(product, high), low);
        
        return add(low, (0,middle), (0,0,high));
    }    
    \end{lstlisting}
\end{algorithm}

\begin{remark}
In der Praxis wählt man den Cutoff nicht so klein wie möglich, um möglichst viel mit der \enquote{besseren} Karatsuba-Methode zu arbeiten, sondern ein kleines bisschen größer, da für sehr kleine Anzahlen von Ziffern die Schul-Methode in der Praxis doch schneller ist als die Karatsuba-Methode. Selbst wenn nur einer der beiden Faktoren wenig Ziffern hat, insbesondere wenn ein Faktor nur eine Ziffer hat, ist die Schul-Methode schneller.

\smallskip
Die Standard-Library von Java nutzt beispielsweise $g=2^{32}$ als Basis, d.h. 32-bit Integer werden direkt per Hardware-Instruktion multipliziert, und $n=80$ als Cutoff\footnote{Diese Zahlen sind Implementierungsdetails, die sich von einer Java-Version zur nächsten auch mal ändern können. Ich habe sie im OpenJDK 21 geprüft. Wer selbst nachschauen will: Es geht um die Klasse \texttt{java.math.BigInteger} und ihre Methode \texttt{multiply(BigInteger)}.}: Sobald einer der Faktoren weniger als $80$ Worte groß ist, wird mit der Schul-Methode multipliziert.

Die C-Bibliothek GMP (GNU Multiprecision Arithmetic library) definiert den Cutoff hardware-abhängig zwischen $n=10$ und $34$ Wörtern, z.B. $n=26$ für ARM64 Apple's M1 Prozessor.
\end{remark}

\begin{remark}
Man beachte, dass $(a_1+a_0)(b_1+b_0)$ nicht ein Produkt von zwei Zahlen $<G$, sondern $<2G$ ist. Wir sind also ganz leicht über die Grenze gegangen, bis zu der wir gehen wollten. Man kann auch nicht einfach $G$ etwas größer zu wählen, sodass die führenden Ziffern $a_1$ und $b_1$ kleiner werden, z.B. wird das mit $A=199999$ nicht funktionieren: Egal, welche (sinnvolle) Potenz $G=10^k$ wir wählen, die Ziffer $a_0$ wird immer $9..99$ sein und somit bei Addition zum Overflow führen.

\smallskip
Da man i.d.R. nicht mit $g=2$ arbeitet (sondern eher $g=2^{32}$), benötigt man für die Summen $a_1+a_0$ und $b_1+b_0$ selbst im Extremfall nur eine weitere Ziffer. Man verliert also in der Praxis nicht wirklich viel. Insbesondere dann, wenn man die Karatsuba-Methode sowieso nur für die größeren Produkte einsetzt; ob man von 1000 zu 500 oder 501 Ziffern reduziert, ist nicht so furchtbar relevant.

\medskip
Wenn man es unbedingt möchte, ist es aber möglich, stattdessen
\[c_1=-(a_1-a_0)(b_1-b_0)+a_1 b_1+a_0 b_0\]
zu verwenden, denn da $0\leq a_0,a_1<G$ gilt, wird auch stets $\abs{a_1-a_0}<G$ sein. Man zahlt dafür den Preis, ein bisschen mehr Buchhaltung mit den Vorzeichen betreiben zu müssen.
\end{remark}

\begin{proposition}
    Algorithmus~\ref{alg:karatsuba} ist korrekt, benötigt $O(n^{\log(3) / \log(2)})$ Multiplikationen, Additionen und Subtraktionen, sowie $O(n)$ Speicher.
\end{proposition}
\begin{proof}
Die Abschätzungen ergeben sich direkt aus dem Mastertheorem (siehe Aufgabe \ref{ex:mastertheorem}), denn die Anzahl der Multiplikationen erfüllt die Rekursion $M(n) = 3 M(\frac{n}{2})$, die Anzahl der Additionen/Subtraktionen $A(n) = 3 A(\frac{n}{2})+2\frac{n}{2} + 3\cdot 2n$.

\medskip
Der Speicherbedarf ergibt sich daraus, dass wir für jede Stufe der Rekursion Speicher für \texttt{high}, \texttt{low}, \texttt{mixed\_a}, \texttt{mixed\_b}, \texttt{product},  \texttt{middle} sowie das Endergebnis benötigen, die jeweils $n$, $n$, $\frac{n}{2}+1$, $\frac{n}{2}+1$, $n+1$, $n$, $2n$ groß sind. Wir runden das zu $7n+O(1)$.\footnote{In der Tat kann man clever sein: Für \texttt{high}, \texttt{low} und \texttt{middle} benötigt man keinen zusätzlichen Speicher, wenn man \texttt{high}, \text{low} direkt im Ergebnis speichert und die Subtraktionen bzw. Addition jeweils in-place ausführt. Dann ist der konstante Faktor 5 statt 7.}

\smallskip
Wenn wir sequentiell ausführen, also jeweils nur einen Zweig im Rekursionsbaum gleichzeitig im Speicher halten müssen, haben also demnach einen Speicherbedarf von 
\[\left(7n + O(1)\right) + \left(7\frac{n}{2} + O(1)\right) + \left(7 \frac{n}{4} + O(1)\right) + \ldots = 2\cdot 7n+O(\log(n)) \qedhere\]
\end{proof}

\begin{remark}
Solange wir nur endlich viele Prozessoren haben und nicht maximal parallel arbeiten, benötigen wir auch nur ein konstantes Vielfaches der sequentiellen Speichermenge.

Wenn wir maximal parallelisieren und somit alle Zweige der Rekursion gleichzeitig im Speicher halten müssen, dann erfüllt der Speicherbedarf auch eine Rekursion wie $S(n) = 3S(\frac{n}{2})+5n$, womit sich dann ein Speicherbedarf von $O(n^{\log(3)/\log(2)})$ ergibt.

In der Praxis hat keine Maschine unendlich viele Prozessoren, sodass man niemals voll parallelisieren kann. Andererseits stoßen wir in der Praxis auch nicht auf Zahlen mit beliebig vielen Ziffern, sondern ausschließlich $\leq 2^{32}$ Ziffern (wenn die Zifferngröße $g=2^{32}$ ist, dann wäre eine einzige solche Zahl schon $\approx \SI{4}{\giga\byte}$ groß!), sodass die Rekursionstiefe $\log(n)$ als praktisch konstant und $\leq 32$ betrachtet werden kann und somit auch in der Nähe einer realistischen Anzahl an Prozessoren liegt.
\end{remark}
% !TeX root = main.tex
% !TeX spellcheck = de_DE

\begin{sheet}

    \begin{problem}[title={Mastertheorem}]
        \label{ex:mastertheorem}
        Beweise das \emph{Mastertheorem für divide-and-conquer-Algorithmen}:

        Gegeben sei eine Rekursion der Form
        \[T(n) = a T\left(\frac{n}{b}\right) + f(n)\]
        für Zahlen $a,b>1$ und eine nichtnegative Funktion $f$. Definiere $c_0 := \frac{\log(a)}{\log(b)}$. Zeige:
        \begin{subproblem}[difficulty={leicht}]
            Wenn $f\in O(n^c)$ mit $c<c_0$, dann ist $T\in\Theta(n^{c_0})$. Speziell für $f(n) := d n^c$ ist sogar $T(n) = k_0 n^{c_0} + k n^c$ mit geeigneten Konstanten $k_0,k$.
        \end{subproblem}
        \begin{subproblem}[difficulty={leicht}]
            Wenn $f\in O(n^{c_0}\log(n)^k)$ mit $k\geq 0$, dann ist $T\in O(n^{c_0} \log(n)^{k+1})$.
        \end{subproblem}
        \begin{subproblem}[difficulty={mittel}]
            Wenn $f\in o(n^c)$ mit $c>c_0$, dann ist $T\in o(n^c)$. Wenn zusätzlich $a f(\frac{n}{b})\leq k f(n)$ für eine Konstante $0\leq k<1$ und alle hinreichend großen $n$ gilt, dann ist $T\in\Theta(f)$.
        \end{subproblem}
    \end{problem}

    \begin{problem}
        Wende das Mastertheorem auf das Beispiel von Sortieralgorithmen an: Viele Sortieralgorithmen wie Heapsort, Mergesort, Quicksort, \ldots arbeiten nach folgendem Prinzip:
        \begin{enumerate}
            \item Teile die zu sortierende Liste in zwei Hälften
            \item sortiere die beiden Hälften mittels rekursiver Anwendung des gleichen Algorithmus
            \item Iteriere einmal über beide (nun sortierte) Teillisten um eine sortierte Gesamtliste zu erstellen.
        \end{enumerate}
        Zeige, dass alle diese Sortieralgorithmen Zeitkomplexität $O(n\log(n))$ haben.
    \end{problem}

    \begin{problem}[title={Strassens Matrixmultiplikationsalgorithmus}]
        Analog zu Karatsubas Trick für Multiplikation ganzer Zahlen gibt es Strassens Trick für Multiplikation von $2\times 2$-Matrizen $A,B$: Wenn wir das Produkt mit $X$ bezeichnen, also
        \[ \begin{pmatrix}
               X_{11} & X_{12} \\ X_{21} & X_{22}
        \end{pmatrix} :=
        \begin{pmatrix}
            A_{11} & A_{12} \\ A_{21} & A_{22}
        \end{pmatrix} \cdot
        \begin{pmatrix}
            B_{11} & B_{12} \\ B_{21} & B_{22}
        \end{pmatrix} = \begin{pmatrix}
                            A_{11} B_{11}+ A_{12} B_{21}  & A_{11} B_{12} + A_{12} B_{22} \\
                            A_{21} B_{11} + A_{22} B_{21} & A_{21} B_{12} +A_{22} B_{22}
        \end{pmatrix}\]
        dann gilt
        \[\begin{pmatrix}
              X_{11} & X_{12} \\ X_{21} & X_{22}
        \end{pmatrix} =
        \begin{pmatrix}
            M_1 + M_4 - M_5 + M_7 & M_3 + M_5             \\
            M_2 + M_4             & M_1 - M_2 + M_3 + M_6
        \end{pmatrix}\]
        wobei
        \begin{align*}
            M_1 &:= (A_{11} + A_{22}) \cdot (B_{11} + B_{22}) \\
            M_2 &:= (A_{21} + A_{22}) \cdot B_{11} \\
            M_3 &:= A_{11} \cdot (B_{12} - B_{22}) \\
            M_4 &:= A_{22} \cdot (B_{21} - B_{11}) \\
            M_5 &:= (A_{11} + A_{12}) \cdot B_{22} \\
            M_6 &:= (A_{21} - A_{11}) \cdot (B_{11} + B_{12}) \\
            M_7 &:= (A_{12} - A_{22}) \cdot (B_{21} + B_{22}) \\
        \end{align*}
        Daraus ergibt sich ein divide-and-conquer-Algorithmus, naheliegenderweise \enquote{Strassen-Algorithmus} genannt. Bestimme die Laufzeit- und Speicherkomplexität dieses Algorithmus.
    \end{problem}

    \begin{remark}
        Um $n\times n$-Matrizen zu multiplizieren genügen also $7$ Multiplikationen von $\frac{n}{2}\times\frac{n}{2}$-Matrizen. Indem man Strassens Trick zweimal anwendet, kann man also auch $7^2=49$ Multiplikationen von $\frac{n}{4}\times\frac{n}{4}$-Matrizen verwenden.

        Google hat nur wenige Tage vor unserem Kurs bekannt gegeben, dass deren AI \enquote{AlphaEvolve} ein gewisses Optimierungsproblem lösen konnte und dabei zu dem Ergebnis gekommen ist, dass man mit nur $48$ Multiplikationen auskommt, was einen leicht besseren Algorithmus als Strassens ergibt; der Exponent ist ca. 0,53\% kleiner. \cite{googleAlphaEvolve}
    \end{remark}

\end{sheet}


\section{Level 2: \texorpdfstring{$O(n^{1+\varepsilon})$}{O(n\^{}(1+ε))} Toom-Cook}\label{toom-cook}
% !TeX root = main.tex
% !TeX spellcheck = de_DE

\begin{remark}
    Der nächste große Durchbruch kam schnell nach Karatsuba: 1963 bereits beschrieb Toom\footnote{Andrei Leonovich Toom, 1942--2022, russischer Mathematiker} eine Verallgemeinerung von Karatsubas Algorithmus; 1966 beschrieb Cook\footnote{Stephen Arthur Cook, geb. 1939, amerikanisch-kanadischer Mathematiker} eine vereinfachte Version, die heutzutage Toom-Cook-Algorithmus genannt wird.

    \medskip
    \enquote{Der} Toom-Cook-Algorithmus ist tatsächlich eine Familie von Algorithmus, parametrisiert durch eine natürliche Zahl $k\geq 2$. In der Praxis werden jedoch einige wenige kleine Werte von $k$ verwendet. Manchmal wird nur der Fall $k=3$ als \enquote{der} Toom-Cook-Algorithmus bezeichnet.

    \smallskip
    Die grundlegende Idee ist es, die Input-Zahlen nicht in 2, sondern $k$ Teile zu zerlegen und die $G$-Ziffern der Inputs
    \[A = a_{k-1} G^{k-1} + \ldots + a_1 G + a_0, \quad B = b_{k-1} G^{k-1} + \ldots + b_1 G + b_0\]
    als Koeffizienten von zwei Polynomen
    \[\alpha(X) = a_{k-1} X^{k-1} + \ldots + a_1 X + a_0, \quad \beta(X) = b_{k-1} X^{k-1} + \ldots + b_1 X + b_0\]
    aufzufassen. Die Zahlen $A$ und $B$ sind Auswertungen dieser Produkte an der Stelle $X=G$; das Produkt $C:=A\cdot B$ ist entsprechend die Auswertung des Polynoms $\gamma(X):=\alpha(X)\beta(X)$ an der Stelle $X=G$. Wenn wir die Koeffizienten von $\gamma$ aus den Koeffizienten von $\alpha$ und $\beta$ berechnen können, dann ist die Ermittlung von $C$ sehr einfach.

    \smallskip
    Das Produkt $\alpha(X)\beta(X)$ mittels der Definition
    \[\alpha(X)\beta(X) = \sum_{i,j} a_i b_j X^{i+j}\]
    auszurechnen, würde uns direkt wieder zur Schulmultiplikation führen. Die entscheidende Einsicht ist aber, dass ein Polynom vom Grad $d$ bereits eindeutig durch seine Werte an $d+1$ beliebigen Stützstellen festgelegt ist. Wir benötigen $2k-1$ Stützstellen für $\gamma$.

    Wenn wir an Stelle der relativ großen Zahl $G$ sehr kleine Stützstellen, z.B.
    \[s\in\set{-k-1,\ldots,-1,0,1,\ldots,k-1}\]
    verwenden, dann sind die Ergebnisse $\alpha(s)$ und $\beta(s)$ eher in der Größenordnung $const\cdot G$, nicht $G^k$.

    Wir können also die Produkte $\alpha(s)\beta(s)$ mit einer kleineren Multiplikation bestimmen und so einen rekursiven Multiplikationsalgorithmus bauen, wenn wir es schaffen, aus den Stützstellen $\gamma(s)$ relativ schnell wieder die Koeffizient von $\gamma$ zu bestimmen.
\end{remark}

\begin{lemma}[Vandermonde-Matrix]
    Evaluation an $q$ Stützstellen $s_0,s_1,\ldots,s_q\in\IR$ ist eine lineare Abbildung $\IR[X]_{\leq d} \to \IR^{q+1}$
    \[\begin{pmatrix}
          \alpha(s_0) \\ \alpha(s_1) \\ \vdots \\ \alpha(s_q)
    \end{pmatrix} = \underbrace{\begin{pmatrix}
                                    s_0^0  & s_0^1  & \cdots & s_0^d  \\
                                    s_1^0  & s_1^1  & \cdots & s_1^d  \\
                                    \vdots & \vdots & \ddots & \vdots \\
                                    s_q^0  & s_q^1  & \cdots & s_q^d
    \end{pmatrix}}_{=:V \in \IR^{(q+1)\times(d+1)}}\cdot\begin{pmatrix}
                                                            \alpha_0 \\ \alpha_1 \\ \vdots \\ \alpha_d
    \end{pmatrix}\]
    Für $q=d$ ist die Matrix $V$ invertierbar.
\end{lemma}
\begin{proof}
    Übung.
\end{proof}

\begin{algorithm}[Toom-Cook-$k$]
    \label{ag:toom_cook}
    Parameter: $k\in\IN_{\geq{2}}$, paarweise verschiedene Stützstellen $\set{s_\ell | 0\leq\ell<2k-1}\subset\IQ$.

    Vorberechnung:
    \begin{itemize}
        \item Die Evaluationsmatrix $\mathcal{E}\in\IQ^{(2k-1)\times k}$ mit $\mathcal{E}_{\ell,m} = s_\ell^m$ für $0\leq\ell<2k-1, 0\leq m<k$.
        \item Die Interpolationsmatrix $\mathcal{I}\in\IQ^{(2k-1)\times(2k-1)}$ mit $(\mathcal{I}^{-1})_{\ell,m} = s_\ell^m$ mit $0\leq\ell,m < 2k$.
    \end{itemize}

    %! suppress = MissingLabel
    \begin{lstlisting}
List<Digit> multiply(List<Digit> a, List<Digit> b) {
    n := max(a.length, b.length);
    if(n < CUTOFF){
        return slow_multiplication(a,b); // algorithms A1 and/or A3
    }
    n_prime := ceil(n/k); // n' is the size of the individual chunks

    for(int m:=0; m<k; m++) {
        // not an actual computation, only a naming convention
        alpha[m] := a[m*n_prime .. (m+1)n_prime-1];
        beta[m]  := b[m*n_prime .. (m+1)n_prime-1];
    }

    // Step 1: evaluate alpha and beta at the chosen points
    eval_alpha := new List<>(2k-1); // each list entry is itself
    eval_beta  := new List<>(2k-1); // a list of n'+O(1) digits

    // Matrix multiplication E*alpha and E*beta
    for(int l:=0; l<2k-1; l++) {
        for(int m:=0; m<k; m++) {
            // multiplications with small, known constants
            // => no recursion necessary here
            eval_alpha[l] := add(eval_alpha[l],
                                 multiply(E[l][m], alpha[m]));
            eval_beta[l]  := add(eval_beta[l],
                                 multiply(E[l][m], beta[m]));
        }
    }

    // Step 2: pointwise multiplication

    // each of the entries if itself a list of 2n'+O(1) digits
    eval_gamma := new List<>(2k-1);

    for(int l:=0; l<2k-1; l++) {
        // recursion here
        eval_gamma[l] := multiply(eval_alpha[l], eval_beta[l]);
    }

    // Step 3: recover coefficients of gamma
    gamma := new List<Digit>(2k-1);

    // Matrix multiplication I*gamma
    for(int l:=0; l<2k-1; l++) {
        for(int m:=0; m<2k-1; m++) {
            // multiplication with constant rationals, i.e.
            // multiplication and division with small, known constants
            // => no recursion necessary here
            gamma[l] := add(gamma[l],
                            multiply(I[l][m], eval_gamma[m]));
        }
    }

    // Step 4: evaluate gamma(g^k)
    result := new List<Digit>[2n];
    for(int l=0: l<2k-1; l++) {
        result[l*n_prime...] := add(result[l*n_prime...], gamma[l]);
    }

    return result;
}
\end{lstlisting}
\end{algorithm}

\begin{proposition}[Toom \cite{toom}, Cook \cite{cook}]
    Der obige Algorithmus ist korrekt, hat Laufzeit $O(n^e)$ mit $e:=\frac{\log(2k-1)}{\log(k)}$ und Speicherbedarf $O(n)$.
\end{proposition}
\begin{proof}
    Analog zum Beweis für den Karatsuba-Algorithmus folgt dies im Wesentlichen aus dem Mastertheorem, da wir die Rekursion $T(n) = (2k-1)T(\frac{n}{k})+O(n)$ für die Laufzeit haben. Dabei benutzen wir, dass Multiplizieren mit kleinen Konstanten Laufzeit $O(n)$ hat.

    Wir benötigen Speicher für \texttt{eval\_alpha}, \texttt{eval\_beta}, \texttt{eval\_gamma}, \texttt{gamma} und \texttt{result}. Das sind viermal $2n+O(1)$ und einmal $2n$ Speicher, die wir benötigen. Das $O(1)$ kommt jeweils davon, dass wir bei den diversen Additionen Platz für einen Übertrag bereitstellen müssen.

    Insgesamt kommen wir also auf $10n+O(1)$ pro Rekursionsstufe. Insgesamt benötigen wir also \[(10n+O(1))\left(1+\frac{1}{k}+\frac{1}{k^2}+\ldots\right) = 10n\frac{k}{k-1}+O(\log(n))\]
    Speicher.
\end{proof}

\begin{remark}
    Wir benötigen $O(k^2)$ viele Multiplikationen/Divisionen mit kleinen Konstanten. Die $O$-Konstante in $O(n)$ für die Multiplikationen/Divisionen hängt aber empfindlich davon ab, welche Konstanten das genau sind. Auch wenn es für die asymptotische Laufzeit nicht relevant ist, muss man sich also doch etwas Mühe geben bei der Wahl der Sützstellen.

    \smallskip
    Besonders einfach ist es, Polynome in $X=0$ auszuwerten, weil dafür gar keine Rechnung notwendig ist, also wählt man $0$ immer als eine der Stützstellen. Ebenfalls kostenlos ist die \enquote{Auswertung bei $X=\infty$}. Damit meint man für ein Polynom vom Grad $d$ die Berechnung von $\lim_{x\to\infty} \frac{p(x)}{x^d}$, was genau den führenden Koeffizienten des Polynoms ergibt. In unserer Notation also $a_{k-1}$, $b_{k-1}$ bzw. $\gamma_{2k-2}$. Man kann sich leicht überlegen, dass $\gamma(\infty)=\alpha(\infty)\beta(\infty)$ ebenfalls erfüllt ist, sodass unsere Rechnung trotzdem funktionieren.
\end{remark}

\begin{remark}
    Für $k=2$ erhält man Karatsubas Algorithmus als Spezialfall, indem man die Stützstellen $0,1,\infty$ wählt. Das ergibt nämlich genau die Produkte
    \[\gamma(0) = \alpha(0)\beta(0) = a_0 b_0\]
    \[\gamma(1) = \alpha(1)\beta(1) = (a_1 1^1+a_0 1^0)(b_1 1^1+b_0 1^0)\]
    \[\gamma(\infty) = \alpha(\infty)\beta(\infty) = a_1 b_1\]
    die wir für Karatsubas Algorithmus benutzt haben.
\end{remark}

\begin{remark}
    Für $k=3$ erhalten wir einen Algorithmus in $O(n^{\log(5)/\log(3)}) \subseteq O(n^{1.465})$.
\end{remark}

\begin{remark}[Große $k$]
    Für $k\to\infty$ ist $\frac{\log(2k-1)}{\log(k)} \searrow 1$. Durch hinreichend groß gewähltes $k$ können wir also für jedes $\varepsilon>0$ einen Multiplikationsalgorithmus in $O(n^{1+\varepsilon})$ hinschreiben.

    \medskip
    In der Praxis verwendet man aber keine großen Werte für $k$, da der Overhead extrem schnell anwächst, während der Break-even-point, ab dem die theoretische asymptotische Laufzeit die tatsächliche Laufzeit bestimmt, viel zu schnell wächst: Wenn wir $k$ so groß genug wählen wollen, damit $\frac{\log(2k-1)}{\log(k)} \approx \frac{\log(2k)}{\log(k)} \leq 1+\frac{1}{e}$ erfüllt ist, dann müssen wir in der Tat $k\geq 2^e$ wählen.

    Wenn wir das aber tun, dann ist der Overhead für einen einzigen Rekursionsschritt in der Größenordnung $C k^2 n = C 2^{2e}n$. Für extrem große $n$ wird das zwar dominiert von $C' n^{1+1/e}$, aber tatsächlich geschieht das erst bei $n\geq \left(\frac{C}{C'}\right)^e 2^{2 e^2}$.

    \smallskip
    Selbst wenn wir nur $O(n^{1.1})$ erreichen wollen (also $e=10$), dann müssen wir $k=2^{10}$ und Zahlen mit vielen hundert Bit betrachten, bevor wir überhaupt eine \emph{Chance} haben, dass der Toom-Cook-$k$-Algorithmus sich tatsächlich auszahlt. Mehr noch: Wenn $k=2^{10}$ ist, dann sind die Potenzen $s_\ell^m$, die wir für die Vandermonde-Matrix ausrechnen müssen, schon im Bereich von bis zu $10 \cdot (2k-1) \approx 20000$ bits groß. Das stellt die Behauptung in Frage, wie \enquote{klein} die kleinen Konstanten eigentlich sind, die da in den drei Matrix-Vektor-Multiplikationen vorkommen. Wir müssen also in Wahrheit eher zehntausende Bits, vielleicht sogar hunderttausende Bits große Zahlen betrachten, bevor sich dieser Algorithmus praktisch auszahlt.
\end{remark}

\begin{remark}
    Nichts desto trotz ist diese Überlegung für die Theorie interessant. Wenn man für jedes $k$ einen geeigneten Cutoff vorberechnet, dann kann man alle Toom-Cook-Algorithmus zu einem gemeinsamen Algorithmus kombinieren, der für hinreichend große Zahlen Toom-Cook-$k$ mit immer größer werdendem $k$ verwendet. Dieser Algorithmus ist dann offenbar asymptotisch schneller als jeder einzelne Toom-Cook-Algorithmus, also in $o(n^{1+\varepsilon})$ für jedes $\varepsilon>0$.

    \smallskip
    Es ist nicht genau bekannt, welche Komplexitätsklasse man mit diesem Ansatz erreichen kann, aber eine Implementierung von Knuth\footnote{Donald Ervin Knuth, geb. 1938, amerikanischer Mathematiker und Informatiker, Autor der \enquote{The Art of Computer Programming} Buchreihe.} erzielt eine Komplexität von $O(n 2^{\sqrt{2 \log_2(n)}} \log(n))$, aber es ist nicht bekannt, ob das die beste Implementierung ist.
\end{remark}

\begin{remark}
    Wegen des schnell zunehmenden Overheads werden nur sehr kleine $k$ praktisch eingesetzt und selbst dann nur für sehr große $n$.

    \smallskip
    Die\footnote{Wieder OpenJDK 21 in diesem Fall; andere Implementierungen der Standardbibliothek könnten theoretisch davon abweichen} Java Implementierung wählt z.B. $n=240$ als Cutoff, d.h. erst wenn beide Faktoren mehr als 240 Worte groß sind, wird die Toom-Cook-$3$-Multiplikation gegenüber der Karatsuba-Methode bevorzugt. Ein höheres $k$ wird gar nicht verwendet in der Implementierung.

    \smallskip
    GMP verwendet $k\leq 8$ in gewissen Grenzen und den (asymptotisch noch schnelleren) Schönhage-Strassen-Algorithmus darüber.
\end{remark}
% !TeX root = main.tex
% !TeX spellcheck = de_DE

\begin{sheet}

\begin{problem}[title={Vandermonde-Determinante}, difficulty={mittel}]
Beweise, dass die Vandermonde\footnote{Alexandre-Théophile Vandermonde, 1735--1796, französischer Mathematiker, Musiker und Chemiker}-Matrix
\[V:=\begin{pmatrix}
s_0^0 & s_0^1 & \cdots & s_0^d \\
s_1^0 & s_1^1 & \cdots & s_1^d \\
\vdots & \vdots & \ddots & \vdots \\
s_d^0 & s_d^1 & \cdots & s_d^d
\end{pmatrix}\]
die Determinante
\[\det(V) = \prod_{0\leq i<j\leq d+1} (s_j-s_i)\]
hat. Insbesondere ist $V$ genau dann invertierbar, wenn die $s_i$ paarweise verschieden sind.
\end{problem}

\begin{problem}[difficulty={leicht}]
Bestimme die Determinante der analogen Matrix für die Stützstellen $s_0,s_1,\ldots,s_{d-1},\infty$.
\end{problem}

\begin{problem}
Verallgemeinere den Toom-Cook-$k$-Algorithmus zum Toom-Cook-$(k_1,k_2)$-Algorithmus, der $A$ in $k_1$ und $B$ in $k_2$ kleinere Stücke zerlegt.
\end{problem}

\begin{remark}
Der Toom-Cook-$(2,3)$-Algorithmus wird manchmal auch als Toom-Cook-$2.5$ bezeichnet.
\end{remark}

\end{sheet}


\section{Level 3: Multiplizieren mittels Fourier-Transformation, Schönhage-Strassen \& Co}\label{fft_multiplication}
% !TeX root = main.tex
% !TeX spellcheck = de_DE

\subsection{Idee}

\begin{remark}
    Wie wir im Zusammenhang mit den Toom-Cook-Algorithmen gesehen haben, kann man Multiplikation großer Zahlen auf Multiplikation von ganzzahligen Polynomen mit kleinen Koeffizienten zurückführen. Und je höher wir den Grad der Polynome wählen, desto effizienter können wir sein, zumindest asymptotisch. Wir haben aber auch gesehen, dass das in Praxis nicht tauglich ist, weil der Overhead quadratisch mit dem Grad der Polynome zu wachsen scheint. Wir können also nicht einfach Grad=Anzahl der Ziffern wählen, weil wir sonst wieder bei einem $O(n^2)$-Algorithmus wären.

    \medbreak
    Ist das also das Ende? Nur, wenn Auswertung eines Polynoms vom Grad $<m$ in $m$ Stützstellen sowie die Interpolation von $m$ Werten zurück zu den Polynomkoeffizienten in $\Theta(m^2)$ sind. Aber wer sagt denn, dass quadratisch die bestmögliche Laufzeit ist?

    In der Tat können wir sehr viel besser auswerten und interpolieren, da wir ja die völlig freie Wahl der Stützstellen haben. Es stellt sich heraus, dass Einheitswurzeln eine besonders eine effiziente Wahl sind, weil die Vandermonde-Matrix dann besonders viel Struktur hat, die man ausnutzen kann, um Rechnungen einzusparen. Das ist die Idee hinter der schnellen Fourier-Transformation (FFT).
\end{remark}

% !TeX root = main.tex
% !TeX spellcheck = de_DE

\subsection{Fast Fourier Transform}\label{sec:fft}

\begin{definition}[DFT]
    Gegeben eine endliche Folge von Koeffizienten $a:=(a_0,a_1,\ldots,a_{n-1})\in\IC^n$ heißt die Folge $\hat{a}:=(\hat{a}_0,\hat{a}_1,\ldots,\hat{a}_{n-1})\in\IC^n$ der Zahlen gegeben durch
    \[\hat{a}_i := \sum_{j=0}^{n-1} a_i \zeta_n^{-ij}\]
    die \udot{diskrete Fourier-Transformation}, kurz \udot{DFT}, von $a$, wobei $\zeta_n\in\IC$ eine primitive $n$-te Einheitswurzel ist, d.h. $\zeta_n^n=1$ und $\zeta_n^i \neq 1$ für $i<n$, beispielsweise $e^{2\pi i / n}$.

    \medskip
    Das umgekehrte Problem, aus $\hat{a}$ wieder $a$ zu rekonstruieren, heißt entsprechend \udot{inverse DFT}.
\end{definition}

\begin{remark}
    Die DFT ist allgemein für unstrukturierte Folgen von Zahlen definiert, weil sie in vielen verschiedenen Zusammenhängen auftritt, in denen diese Zahlenfolgen unterschiedliche Bedeutung und Herkunft haben.

    Für unser Ziel ist es aber zweckmäßig, die Zahlen $a$ als Koeffizienten eines Polynoms
    \[\alpha(X) := a_0 + a_1 X + \ldots + a_{n-1}X^{n-1} \in\IC[X]\]
    aufzufassen. Dann ist nämlich $\hat{a}_i = \alpha(\zeta_n^{-i})$.
\end{remark}

\begin{lemma}
    Es stellt sich heraus, dass die iDFT im Wesentlichen genauso aussieht wie die DFT selbst. Mit den Bezeichnungen von oben gilt:
    \[a_i = \frac{1}{n}\sum_{k=0}^{n-1} \hat{a}_k \zeta_n^{ik}\]
\end{lemma}
\begin{proof}
    Definitionsgeschubse.
\end{proof}

\begin{remark}
    Eben weil es so viele verschiedene Arten gibt, in denen die DFT auftreten kann, gibt es verschiedene Konventionen, die Definition zu schreiben, die aber im Wesentlichen äquivalent sind.

    So kann man z.B.\ das andere Vorzeichen wählen und stattdessen $\alpha$ in $\zeta_n^{+i}$ auszuwerten. Da $\set{1, \zeta_n, \zeta_n^2, \ldots} = \set{1, \zeta_n^{-1}, \zeta_n^{-2}, \ldots}$ ist, führt das nur dazu, dass die Komponenten von $\hat{a}$ in einer anderen Reihenfolge auftreten, aber ansonsten die gleichen Werte haben.

    Manchmal wird der Faktor $\frac{1}{n}$ auch für die DFT anstelle der iDFT verwendet. Beides zusammen führt im Wesentlichen dazu, dass das, was wir jetzt als iDFT bezeichnen, zur DFT erklärt wird und umgekehrt. Da Hin- und Rückrichtung aber sowieso zum Verwechseln ähnlich ist, macht das auch keinen großen Unterschied mehr, welche von den beiden Operationen man als DFT und welche man als iDFT bezeichent.

    Manchmal wird der Faktor einfach gleichmäßig verteilt, sodass in beiden Formeln ein $\frac{1}{\sqrt{n}}$ vorkommt.
\end{remark}

\begin{theorem}[Gauss, Cooley-Tukey uvm.]
    Ist $n$ eine Zweierpotenz, so ist möglich, die DFT eines Vektors $a\in\IC^n$ mit $n\log(n)$ komplexen Additionen und Multiplikationen und $n+O(1)$ Speicherbedarf zu berechnen.
\end{theorem}
\begin{remark}
    Die Annahme, dass $n$ eine Zweierpotenz ist, ist in der Tat überflüssig. Sie ist nur für den konkreten Algorithmus nötig, den wir jetzt besprechen wollen. Es gibt andere Algorithmen, die für andere $n$ funktionieren und ebenfalls in $O(n\log(n))$ sind.
\end{remark}
\begin{proof}
    Wir werden die DFT als Auswertung des Polynoms $\alpha$ in den $n$ Punkten
    \[\zeta_n^0, \zeta_n^1,\zeta_n^2 \ldots, \zeta_n^{n-1}\]
    betrachten.

    \smallskip
    Die erste nützliche Beobachtung ist, dass wir $\alpha$ hier in Wirklichkeit als Element von $\IC[X]/(X^n-1)$ auffassen können, da die Auswertung in $n$-ten Einheitswurzeln eh keinen Unterschied zwischen $X^n-1$ und $0$ feststellen kann, da ja $\zeta^n-1=0$ ist. Der Ring $\IC[X]/(X^n-1)$ wird nun mittels des chinesischen Restsatzes rekursiv in kleinere Ringe (jeweils mit halber Dimension) zerlegt:

    Dabei benutzen wir in jedem Schritt, dass $X^{2k}-z^2 = (X^k-z)(X^k+z)$ gilt und das eine Zerlegung in paarweise teilerfremde Faktoren ist (wenn $z\neq 0$), sodass wir den chinesischen Restsatz tatsächlich anwenden können.

    Da $n$ eine Zweierpotenz, $\zeta_n$ eine $n$-te Einheitswurzel und $1=\zeta_n^n$ ist, kann man per Induktion zeigen, dass man in jedem Schritt Faktoren der Form $X^k-\zeta_n^{ak}$ erhält.

    \medskip
    Am Beispiel von $n=8$ sieht das so aus wie in~\ref{fig:fft:example_n_equals_8} dargestellt.

    \begin{figure}[hpt]
        \begin{tikzcd}[sep=tiny, inner sep=0]
            &&& \IC[X]/(X-\zeta_8^0) \\
            && \IC[X]/(X^2-\underbrace{1}_{=\zeta_8^0}) \ar[ru]\ar[rd] & \\
            &&& \IC[X]/(X-\zeta_8^4) \\
            & \IC[X]/(X^4-\underbrace{1}_{=\zeta_8^0}) \ar[ruu]\ar[rdd] && \\
            &&& \IC[X]/(X-\zeta_8^2) \\
            && \IC[X]/(X^2-\underbrace{(-1)}_{\zeta_8^{2\cdot 2}}) \ar[ru]\ar[rd] & \\
            &&& \IC[X]/(X-\zeta_8^6) \\
            \IC[X]/(X^8-\underbrace{1}_{=\zeta_8^0}) \ar[ruuuu] \ar[rdddd] & & \\
            &&& \IC[X]/(X-\zeta_8) \\
            && \IC[X]/(X^2-\underbrace{i}_{=\zeta_8^2}) \ar[ru]\ar[rd] & \\
            &&& \IC[X]/(X-\zeta_8^5) \\
            & \IC[X]/(X^4-\underbrace{(-1)}_{=\zeta_8^4}) \ar[ruu]\ar[rdd] && \\
            &&& \IC[X]/(X^2-\zeta_8^3) \\
            && \IC[X]/(X^2-\underbrace{(-i)}_{=\zeta_8^{3\cdot 3}}) \ar[ru]\ar[rd] & \\
            &&& \IC[X]/(X^2-\zeta_8^7)
        \end{tikzcd}
        \caption{FFT am Beispiel $n=8$}
        \label{fig:fft:example_n_equals_8}
    \end{figure}

    Man beachte, dass wir am Ende die Reduktionen auf die eindimensionalen Quotienten $\IC[X]/(X - \zeta_n^i)$ erhalten, also die Auswertung im Punkt $\zeta_n^i$, also genau die $i$-te Komponente der DFT\@.

    \medskip
    Die Laufzeit dieses Algorithmus kommt daher, dass die Reduktion von $\IC[X]/(X^k-z^2)$ zu $\IC[X]/(X^{k/2}-z)$ mit $k/2$ komplexen Multiplikationen und Additionen erledigt werden kann:

    Wir schreiben eine Restklasse $[f] \in \IC[X]/(X^k-z^2)$ als Polynom vom Grad $\leq k-1$:
    \[f \equiv f_0 + f_1 X + f_2 X^2 + \ldots f_{k-1}X^{k-1} \mod X^k-z^2\]
    und gruppieren jetzt nach Koeffizienten größergleich bzw. kleiner als $k/2$:
    \begin{align*}
        f \equiv & (f_0     &+ f_1 X      &+ \ldots + f_{k/2-1}X^{k/2-1}) + \\
        & (f_{k/2} &+ f_{k/2+1}X &+ \ldots + f_{k-1}X^{k/2-1})X^{k/2} \mod X^k-z^2
    \end{align*}
    Da wir dieses Polynom nun modulo $X^{k/2}-z$ reduzieren wollen, können wir $X^{k/2}$ durch die Konstante $z$ ersetzen:
    \[f \equiv (f_0+f_{k/2}z) + (f_1+f_{k/2+1}z)X^1+(f_2+f_{k/2+2}z)X^2 + \ldots (f_{k/2}+f_{k-1}z)X^{k/2-1} \mod X^{k/2}-z\]
    Das sind also genau $k/2$ komplexe Multiplikationen und $k/2$ Additionen, die wie für diese Reduktion benötigen.

    \medskip
    In jedem Reduktionsschritt benötigen wir das zweimal: Wir reduzieren von $\IC[X]/(X^k-z^2) = \IC[X]/(X^k-(-z)^2)$ zu $\IC[X]/(X^{k/2}-z)$ und zu $\IC[X]/(X^{k/2}-(-z))$.
    Das machen wir für mehrere $k$, sodass wir am Ende bei genau $n$ Additionen bzw. Multiplikationen in jeder Ebene der Rekursion landen.

    \medskip
    Dass der Speicherbedarf $O(n)$ ist, folgt daraus, dass wir in jedem Reduktionsschritt die Anzahl der Faktoren verdoppeln, aber jeder Faktor nur noch halb so groß ist. Im Beispiel $n=8$ beginnen wir etwa mit einem Vektor der Länge $8$, reduzieren auf zwei Vektoren der Länge $4$, reduzieren auf vier Vektoren der Länge $2$ und schließlich acht Vektoren der Länge $1$. Für jede neue Ebene der Rekursion benötigen wir nur Speicher für genau $n$ neue komplexe Zahlen.

    Wenn wir die $n$ Zahlen der vorherigen Level wegwerfen, nachdem wir sie nicht mehr benötigen, kommen wir mit Speicher für $2n$ Zahlen aus. Wenn wir die Formel genau angucken, sehen wir, dass wir immer die $i$-ten Koeffizienten der beiden Reduktionen aus dem $i$-ten und $(i+k/2)$-ten Koeffizient des Inputs berechnen. Das heißt, dass wir den Input einfach direkt mit dem Ergebnis überschreiben können und mit $n+O(1)$ Speicher auskommen.
\end{proof}
% !TeX root = main.tex
% !TeX spellcheck = de_DE

\subsection{Der Schönhage-Strassen-Algorithmus}\label{schoenhage-strassen}

\begin{corollary}
    Die Multiplikation zweier Zahlen $A,B\in\IZ$ mit $n$ Ziffern kann mit $O(n\log(n))$ arithmetischen Operationen auf komplexen Zahlen und Speicher für $O(n)$ komplexe Zahlen berechnet werden.
\end{corollary}
\begin{proof}
    Mittels FFT ist es jetzt klar wie es geht: Wir nutzen die Ziffern von $A$ und $B$ als Koeffizienten von Polynomen
    \begin{align*}
        \alpha(X) &= a_0 + a_1 X^1 + \ldots a_{n-1} X^{n-1}\\
        \beta(X)  &= b_0 + b_1 X^1 + \ldots b_{n-1} X^{n-1}
    \end{align*}
    Berechnen $\alpha(\zeta_{2n}^i)$ und $\beta(\zeta_{2n}^i)$ für alle $i=0,1,\ldots,2n-1$ in $O(n\log(n))$ mittels FFT. $\gamma(\zeta_n^i) = \alpha(\zeta_n^i)\cdot\beta(\zeta_n^i)$ auszurechnen, kostet noch einmal $2n$ Multiplikationen. Die inverse Fourier-Transformation wird ebenfalls mit FFT in $O(n\log(n))$ ausgerechnet.

    Damit haben wir die Koeffizienten von $\gamma(X)$. Man beachte, dass die Koeffizienten von $\gamma$ exakt ganzzahlig sind, auch wenn die Zwischenergebnisse komplexe Zahlen waren!

    Durch Einsetzen von unserer Zahlenbasis $X=g$ erhalten wir das Produkt $A\cdot B = C=\gamma(g)$. Das geht in $O(n)$ elementaren Operationen auf den Ziffern.
\end{proof}

\begin{remark}
    Haben wir damit das Problem gelöst? Nicht ganz, denn niemand sagt uns, dass \enquote{arithmetische Operationen auf komplexen Zahlen} genauso billig sind wie Operationen auf einzelnen Ziffern, was ja eigentlich unsere Maßeinheit ist.

    \medskip
    Da wir a priori wissen, dass die Koeffizienten $\gamma_0, \gamma_1, \ldots, \gamma_{2n-1}$, die wir berechnen wollen, ganze Zahlen sind, können wir das Rechnen mit unendlicher Präzision durch eine Rechnung mit hinreichend langen Festkomma-Zahlen ersetzen. Die Länge der Festkomma-Zahlen muss so bestimmt werden, dass das Endergebnis einen Fehler $<\frac{1}{2}$ hat, sodass wir durch simples Runden zur nächsten ganzen Zahl das Ergebnis bekommen.

    Sobald wir uns erst einmal auf diese Länge festgelegt haben, ist das Rechnen mit Festkomma-Zahlen wieder nur die Addition und Multiplikation von ganzen Zahlen dieser Länge. Solange wir feststellen, dass die notwendige Anzahl an Ziffern für die FFT und iFFT der Länge $n$ signifikant kleiner als $n$ ist, können wir so einen rekursiven Multiplikationsalgorithmus konstruieren.

    \medskip
    Schönhage\footnote{Arnold Schönhage, dt. Mathematiker, geb. 1934} und Strassen\footnote{Volker Strassen, dt. Mathematiker, geb. 1936} haben genau solche eine Analyse durchgeführt und sind zu folgendem Ergebnis gekommen.
\end{remark}

\begin{theorem}[Schönhage-Strassen I {\citep[286f]{schoenhage_strassen}}]
    Multiplikation von zwei $n$-Bit Zahlen ist in $O(n\log(n)^{1+\varepsilon})$ elementaren arithmetischen Operationen auf Ziffern möglich. Präziser kann man für jedes $k\in\IN$ die Komplexität
    \[O(n\cdot \log(n) \cdot \log(\log(n)) \cdots (\log^k(n))^2)\]
    erreichen.
\end{theorem}

\begin{remark}
    Der FFT-Algorithmus, den wir betrachtet haben, ist nicht fundamental davon abhängig, dass mit komplexen Zahlen gerechnet wird. Man kann sich leicht davon überzeugen, dass man $\IC$ durch jeden kommutativen Ring $R$ ersetzen kann, der
    \begin{enumerate}
        \item eine primitive $2n$-te Einheitswurzel $\zeta$ enthält.
        \item ein multiplikatives Inverses von $n$ enthält.
    \end{enumerate}

    Wenn wir eine Zweierpotenz $n=2^\ell$ verwenden wollen, ist das z.B.\ vom Restklassenring $\IZ/(2^{2^\ell}+1)$ erfüllt, denn $\zeta:=2$ erfüllt per Konstruktion $\zeta^{2^\ell} = -1, \zeta^{2^\ell \cdot 2}=1$ und ist somit eine $2^{\ell+1}$-te primitive Einheitswurzel. Die Restklassen repräsentiert man natürlicherweise durch ganze Zahlen zwischen $0$ und $2^{2^\ell}$, sodass man keine komplexe Arithmetik benötigt, sondern nur ganzzahlige Arithmetik\footnote{Man beachte, dass wegen $2^{2^\ell} \equiv -1$ die notwendigen Divisionen-mit-Rest durch Additionen, Subtraktionen und Bitshifts realisiert werden können}.

    \smallskip
    Natürlich ist $2^{2^\ell}=2^n$ wieder in der Größenordnung, in der unsere Inputs $A$ und $B$ waren, denn die hatten ja $n$ Ziffern. Wenn wir $g=2^{32}$ verwenden, dann hat $2^n$ zwar trotzdem weniger Bits als $g^n$, aber nur um einen konstanten Faktor weniger, sodass wir diesen Ansatz rekursiv anwenden müssen, wenn wir Arithmetik in $\IZ/(2^{2^\ell}+1)$ verwenden wollen.

    Wenn wir sowieso rekursiv arbeiten, dann hindert uns aber niemand daran, wie bei den vorherigen Algorithmen eine passende Zwischengröße $G$ zu wählen, sodass unsere Rekursionsschritte mit einem deutlich kleineren $n$ (und damit einem deutlich kleineren $\ell$) zufrieden sind.

    \medskip
    All diese Überlegungen haben Schönhage und Strassen auch durchgeführt und haben so ihr vorheriges Ergebnis wie folgt verbessert:
\end{remark}

\begin{theorem}[Schönhage-Strassen II {\citep[287ff]{schoenhage_strassen}}]
    Multiplikation von zwei $n$-Bit Zahlen ist in $O(n\log(n)\cdot\log(\log(n)))$ elementaren arithmetischen Operationen auf Ziffern möglich.
\end{theorem}

\begin{remark}
    Beliebige Zahlen $n$ sind natürlich keine Zweierpotenzen. Wir können zwar o.B.d.A.\ annehmen, dass $n$ eine Zweierpotenz ist, indem wir den Zahlen $A$ und $B$ ggf. einfach ein paar Nullen voranstellen, aber im Extremfall verdoppelt das die Länge der Zahlen.

    Für die asymptotische Komplexität ist ein Faktor 2 nicht relevant, für die Praxis aber schon, da der Schönhage-Strassen-Algorithmus ja sowieso erst eingesetzt wird, wenn die Inputs sehr groß sind. \footnote{Cutoff in GMP liegt bei mehreren tausend Wörtern abhängig von der verwendeten Hardware}

    Wenn aber alleine unsere Inputs schon Kilo- und Megabytes groß sind, dann ist selbst ein konstanter Faktor im Speicherbedarf störend.
\end{remark}
% !TeX root = main.tex
% !TeX spellcheck = de_DE

\subsection{Level \texorpdfstring{$\infty$}{infinity}? \texorpdfstring{$O(n\log(n))$}{O(n log(n))} Harvey-Hoeven}

\begin{remark}
    Natürlich stellt sich die unmittelbare Frage, wie weit man dieses Spiel treiben kann und was das tatsächliche Optimum für Multiplikation ist.

    Schönhage \& Strassen stellten folgende Vermutung auf
\end{remark}

\begin{conjecture}[Schönhage-Strassen]
    Multiplikation von $n$-Bit Zahlen ist in $\Theta(n\log(n))$ möglich.
\end{conjecture}

\begin{theorem}[Harvey-Hoeven, 2019]
    Die Schönhage-Strassen-Vermutung ist wahr.
\end{theorem}

\begin{remark}
    Der Beweis ist komplex. Er kombiniert geschickt verschiedene Ideen, um den FFT-Ansatz zu optimieren.
\end{remark}

\begin{remark}
    Eine grundlegende Idee ist, dass eine FFT der Länge $n_1 n_2$ durch eine $n_1\times n_2$ zweidimensionale FFT (oder noch mehr Faktoren) zu ersetzen. Eine zweidimensionale FT von $(a_{ij})$ ist definiert durch
    \[\hat{a}_{ij} := \sum_{k=0}^{n_1-1} \sum_{l=0}^{n_2-1} a_{kl} \zeta_{n_1}^{-ki} \zeta_{n_2}^{-lj}\]
    Analog sind multidimensionale FTs definiert.\footnote{Wer's kennt: Abstrakt formuliert heißt das nur, dass die $n_1\times n_2$-FT das Tensorprodukt der $n_1$-FT und der $n_2$-FT ist. Aus dem Grund übertragen sich alle Algorithem für 1D-Fouriertransformationen auch auf multidimensionale FTs.} Der eindimensionale FFT-Algorithmus überträgt sich in natürlicher Weise auf den multidimensionalen Fall.

    \medskip
    Für unseren Ansatz, FFT zu benutzen um Polynome zu multiplizieren, bedeutet das, dass man nicht nur das $g$ in
    \[A = a_0 + a_1 g + a_2 g^2 + \ldots\]
    durch eine Unbekannte $X$ ersetzt, sondern dass man zusätzlich $g^{n_1}$ durch eine weitere Unbekannte $Y$ ersetzt. Jeder Exponenten $e<n_1 n_2$ kann eindeutig als $e'+e'' n_1$ mit $e'<n_1, e''<n_2$ geschrieben werden und so wird $g^e$ durch $X^{e'} Y^{e''}$ ersetzt. So erhält man statt eines Polynoms in einer Variablen ein Polynom in zwei (oder noch mehr) Variablen, dessen Exponenten durch $n_1$ bzw. $n_2$ beschränkt sind.

    \medskip
    Natürlich ist nicht jede Zahl $n$ nichttrivial als $n_1 n_2$ faktorisierbar. Da wir aber jederzeit $A$ und $B$ eine Null voranstellen können, können wir jederzeit zu $n+1$ übergehen und so eine zusammengesetzte Zahl erhalten, wenn $n$ zufällig eine Primzahl war. Wenn wir uns wenige (!) Nullen mehr erlauben, können wir $n+k$ so wählen, dass möglichst viele möglichst kleine Faktoren vorkommen und somit maximal zerlegt werden kann.

    \medskip
    Eine zweidimensionale $n_1\times n_2$ FFT benötigt zwar trotzdem $O(n\log(n))$ Operationen, aber da wir nur $n_1$-te und $n_2$-te Einheitswurzeln benötigen anstatt $n$-te Einheitswurzeln, können wir mit deutlich kleineren Restklassenringen arbeiten und einschränken, wie oft der zu findende Multiplikationsalgorithmus rekursiv angewendet werden muss, bis die auftretenden Koeffizienten klein genug sind.
\end{remark}

\begin{remark}
    Speziell für Primzahlen gibt es außerdem \enquote{Rader's trick}, der es erlaubt eine FFT der Länge $p$ auf eine FFT der Länge $p-1$ zurückzuführen. Dies ermöglicht weitere Freiheitsgrade bei der Wahl möglichst gut teilbarer Zahlen.
\end{remark}

\begin{remark}
    Wenn man in der Situation ist, eine multidimensionale $n_1\times n_2\times \cdots\times n_d$ FFT mit Zweierpotenzen $n_1\leq n_2\leq \ldots\leq n_d$ durchzuführen, dann kann man in jeder Dimension außer der letzten die FFT in $O(n_i)$ statt $O(n_i \log(n_i))$ durchführen, indem man anstatt einer komplexen Zahl (oder einem Element eines Restklassenrings) als Einheitswurzel die Unbekannte $X_d \in R[X_d]/(X_d^{n_d}-1)$ verwendet, die ja ebenfalls eine $n_d$-te Einheitswurzel ist. Das Multiplizieren mit einer Unbekannten ist in Polynomringen völlig gratis, da es nur die Koeffizienten herum permutiert.

    Wenn wir die $n_i$ alle in der Größenordnung $n^{1/d}$ wählen können, dann reduziert das den Aufwand für die FFTs von $\approx n\log(n)$ zu $\approx \frac{n\log(n)}{d}$.
\end{remark}

\begin{remark}
    Harvey und Hoeven konnten zeigen, dass es immer genügend viele Zahlen \enquote{in der Nähe von $n$} gibt, dass man genügend schnell zu einer Zahl mit sehr vielen sehr kleinen Faktoren kommen kann und diese Faktoren klein genug sind.
\end{remark}

\begin{remark}
    Die Details sind sehr alle technisch und der Algorithmus, der dabei herauskommt, ist so komplex und hat so viel Overhead, dass er nur auf dem Papier interessant ist. Der Overhead ist so gigantisch, dass kein praktisches Problem groß genug ist, damit der Harvey-Hoeven-Algorithmus den Schönhage-Strassen-Algorithmus tatsächlich schlägt.

    \smallskip
    Harvey und Hoeven selber schätzen, dass man Zahlen mit $\approx 10^{14}$ Ziffern benötigt, bevor sich ihr Algorithmus auszahlt.

    \medskip
    Während Schönhage-Strassen tatsächlich in Bibliotheken wie GMP implementiert ist und auch praktisch verwendet wird, ist mir (zum jetzigen Zeitpunkt) keine einzige Implementierung von Harvey-Hoeven bekannt.
\end{remark}
% !TeX root = main.tex
% !TeX spellcheck = de_DE

\begin{sheet}
    \begin{problem}[title={Eine interessante Permutation}]
        In unserem FFT-Algorithmus haben wir im Beispiel $n=8$ mit Zahlen $a_0,a_1,\ldots,a_7$ in dieser Reihenfolge begonnen, aber am Ende den fourier-transformierten Vektor in der Reihenfolge $\hat{a}_0, \hat{a}_4, \hat{a}_2, \hat{a}_6, \hat{a}_1,\hat{a}_5, \hat{a}_3, \hat{a}_7$ erhalten.

        Beschreibe die Permutation $\sigma:\set{0,1,\ldots,n-1} \to \set{0,1,\ldots,n-1}$, die dadurch bestimmt ist, dass $\hat{a}_{\sigma(0)}, \hat{a}_{\sigma(1)}, \hat{a}_{\sigma(2)}, \ldots$ genau die Reihenfolge ist, in der der fourier-transformierte Vektor ausgegeben wird.
    \end{problem}

    \begin{problem}[title={FFT ist numerisch stabil}]
        Zeige, dass im Cooley-Tukey-Algorithmus $O(\log(n))$ Nachkommastellen an Genauigkeit verloren gehen:

        Es sei $n=2^\ell$ eine Zweierpotenz, $(a_0,\ldots,a_{n-1})\in\IC^n$ der Vektor, den wir fourier-transformieren wollen und $(b_0, \ldots, b_{n-1})\in\IC^n$ eine Näherung an $a$ mit $\abs{a_j-b_j}\leq 2^{-m}$. OBdA nehmen wir $\abs{a_j}, \abs{b_j} < 1$ an.

        \begin{subproblem}
            Zeige, dass für die fourier-transformierten Vektoren $(\hat{a}_0,\ldots,\hat{a}_{n-1})$ bzw. $(\hat{b}_0,\ldots,\hat{b}_{n-1})$ gilt:
            \[\abs{\hat{a}_j - \hat{b}_j} \leq 2^{-m+c}\]
            mit einer Konstanten $c\geq 1$, d.h.\ pro Rekursionsstufe gehen $c$ viele Bits an Genauigkeit verloren.
        \end{subproblem}
        \begin{subproblem}
            Natürlich hängt das nicht von der Wahl der Basis $2$ und ist genauso wahr bezüglich jeder anderen Basis $g$.
        \end{subproblem}
        \begin{subproblem}
            Zeige, dass analoges gilt, wenn die Fouriertransformation von $b$ nicht mit den exakten Einheitswurzeln $\zeta_n^0,\zeta_n^1,\ldots$ ausgerechnet wird, sondern stattdessen mit Näherungen $z_j \approx \zeta_n^j$ gerechnet wird.

            Hinweis: Die Näherungen $z_j$ können bessere Näherungen sein, als die $b_j$ es sind.
        \end{subproblem}
        \begin{subproblem}
            Zeige, dass das alles genauso auch für die inverse FFT gilt.
        \end{subproblem}
    \end{problem}

    \begin{problem}
        Es sei $n=2^\ell$ eine Zweierpotenz und $m\in\IN$ beliebig. Finde einen Algorithmus, um $\zeta_n:=e^{2\pi i/n}$ auf $m$ Nachkommastellen genau zu bestimmen. Zeige, dass man mit $O(m\log(n))$ elementaren Operationen mit komplexen Zahlen auskommen kann?

        \smallskip
        Hinweis 1: Um von $\ell$ zu $\ell+1$ zu kommen, muss man eine (bestimmte von den zwei) Quadratwurzel(n) einer komplexen Zahl berechnen.

        Hinweis 2: Quadratwurzeln rechnet man mit Newton-Iteration aus. Starre so lange auf Abbildung~\ref{fig:fft:approximation_zeta_8}, bis dir klar wird, wie man das macht.

        \begin{figure}[ht]
            \centering
            \begin{tikzpicture}[
                dot/.style={draw,fill,circle,inner sep=1pt}
            ]
                \draw[step=0.5,gray,very thin,dotted] (-1.4,-1.4) grid (1.4,1.4);

                \draw[->,gray] (-2,0) -- (2,0) node[below] {Re};
                \draw[->,gray] (0,-2) -- (0,2) node[left] {Im};

                \draw[help lines] (0,0) circle (1);

                \foreach \i in {0,...,7} {
                    \node[dot, label={\i*360/8:$\zeta_8^{\i}$}] (zeta\i) at (\i*360/8:1) {};
                }

                \draw[-,thin] (zeta0) -- (zeta2);
                \node[dot] (zeta1approx) at (0.5,0.5) {};
            \end{tikzpicture}
            \caption{Eine Approximation an $\zeta_8$ ausgehend von $\zeta_4$}
            \label{fig:fft:approximation_zeta_8}
        \end{figure}
    \end{problem}

    \begin{problem}[title={Schönhage-Strassen I}]
        \begin{subproblem}
            Folgere, dass man mit $O(\log(n))$ Bits Genauigkeit rechnen muss, um ganze Zahlen mit $n$ Bits mittels komplexer FFT zu multiplizieren.
        \end{subproblem}
        \begin{subproblem}
            Folgere daraus den ersten Satz von Schönhage-Strassen, dass man ganze Zahlen in $O(n\log(n)^2)$ elementaren Operationen multiplizieren kann.

            Überzeuge dich insbesondere davon, dass man auch innerhalb von $O(n\log(n))$ die für die FFT benötigten Einheitswurzeln auf $O(\log(n))$ Nachkommastellen genau berechnen kann.
        \end{subproblem}
        \begin{subproblem}
            Folgere daraus wiederum per Induktion, dass es auch in
            \[O(n\cdot\log(n)\cdot \log(\log(n)) \cdots (\log^k(n))^2)\]
            geht.
        \end{subproblem}
    \end{problem}
\end{sheet}

\pagebreak
\setcounter{section}{98}


\section{Anhang: Gruppen \& Ringe}\label{algebra}
% !TeX root = main.tex
% !TeX spellcheck = de_DE

\subsection{Gruppen}\label{sec:groups}

\begin{definition}[Gruppen]
    \label{gruppen:def}
    Eine \udot{Gruppe} $(G,*)$ besteht aus
    \begin{itemize}
        \item einer Menge $G$ und
        \item einer Abbildung $*: G\times G, (g,h) \mapsto g\cdot h$, genannt \emph{Multiplikation},
    \end{itemize}
    die die \enquote{Gruppen-Axiome} erfüllen:

    \begin{description}
        \item{(G1)} Assoziativität: $\forall x,y,z\in G: x\ast (y\ast z)=(x\ast y)\ast z$.
        \item{(G2)} Neutrales Element: $\exists e \in G \forall x\in G: x\ast e=e\ast x=x$.
        \item{(G3)} Inverse Elemente: $\forall x \in G \exists x' \in G: x\ast x' = e = x'\ast x$.
    \end{description}
    Die Gruppe heißt \udot{abelsch}\footnote{Niels Hendrik Abel, norwegischer Mathematiker, 1802--1829} oder \udot{kommutativ}, falls zusätzlich dazu die folgende Bedingung erfüllt ist:
    \begin{description}
        \item{(G4)} Kommutativität: $\forall x,y,\in G: x\ast y=y\ast x$.
    \end{description}

    Der Kürze halber schreibt man oft nur $G$ statt $(G,\ast)$, falls klar ist, welche Operation $\ast$ gemeint ist.
\end{definition}

\begin{remark}
    Es gibt zwei Klassen von Gruppen. Die einen sind Gruppen, bei denen man traditionell eine multiplikative Schreibweise wählt, d.h.\ der Name der Gruppenverknüpfung $\ast$ erinnert an ein Multiplikationssymbol, etwa $*$, $\cdot$, $\otimes$, $\circ$ etc. Oder sogar, wie bei der gewöhnlichen Multiplikation: Man lässt das Symbol ganz weg und schreibt nur noch $xy$. Praktisch alle nichtabelschen Gruppen schreibt man multiplikativ. In dieser Schreibweise nennt man das (es gibt wirklich nur eines, siehe Lemma~\ref{gruppen:eindeutig_1}) neutrale Element dann $1$ statt $e$ und das (ebenfalls eindeutige, siehe Lemma~\ref{gruppen:eindeutig_inv}) von $x$ nennt man $x^{-1}$ statt $x'$.

    Die zweite Klasse sind Gruppen, die man traditionell additiv schreibt, d.h.\ der Name der Gruppenverknüpfung erinnert an ein Additionssymbol, etwa $+$ oder $\oplus$. Dies wird fast ausschließlich bei abelschen Gruppen angewandt. In dieser Schreibweise nennt man das neutrale Element dann $0$ statt $e$ und das inverse Element von $x$ nennt man $-x$ statt $x'$.

    Es gibt aber natürlich keinen inhaltlichen Unterschied zwischen diesen Schreibweisen. Die Wahl, wie wir etwas aufschreiben, ist ja nur eine Frage der Ästhetik, sie hat keine inhaltlichen Konsequenzen.
\end{remark}

\begin{example}
    \begin{enumerate}
        \item $G = \IZ$ mit $x*y = x+y$. Es ist $e=0$, $x'=-x$. Diese Gruppe ist abelsch. Analog sind auch $(\IQ,+)$, $(\IR,+)$ und $(\IC,+)$ abelsche Gruppen.
        \item $G = \IR_{>0} = \Set{x \in \IR | x > 0 }$ mit $x*y = xy$ (Multiplikation). Es ist $e=1$, $x' = \frac{1}{x}$.
        \item Die \udot{triviale} Gruppe $G = \set{1}$, mit $1*1 = 1$. Neutrales Element $1$, und $1' = 1$.
        \item Die Menge $G = \set{+1,-1}$, mit Multiplikation. Es ist $e=1$, $x'=\frac{1}{x}=x$. Dies ist eine endliche Gruppe: $\abs{G} = 2$.
        \item Die Matrizengruppen $GL_n(K)$, $SL_n(K)$, $O_n(\IR)$, $SO_n(\IR)$, $U_n(\IC)$, $SU_n(\IC)$, \ldots sind wichtig für Geometrie und Physik:

        $O_n(\IR)$ beschreibt Drehungen und Spiegelungen im $n$-dimensionalen Raum; $SO_n(\IR)$ beschreibt nur die Drehungen. $U_1(\IC)$ z.B.\ beschreibt die Phasenverschiebung zwischen zwei rotierenden Systemen; $SU_2(\IC)$ ist z.B.\ für die Beschreibung der elektromagnetischen Kraft zuständig und ist u.A.\ für das quantenphysikalische Phänomen des Spins verantwortlich; $SU_3(\IC)$ spielt in der Beschreibung der starken Kernkraft eine Rolle.

        Andere Gruppen wie z.B. $O_{3,1}(\IR)$ kommen in der Relativitätstheorie vor (Sie erkennen drei Raum- und eine Zeitdimension). Diese Gruppe beschreibt sogenannte \enquote{Lorentz-Boosts}.

        Diese Gruppen sind fast alle nicht abelsch und unendlich.
    \end{enumerate}
\end{example}

\begin{example}
    \begin{enumerate}
        \item $(\IN,+)$ ist hingegen keine Gruppe. (G1) und (G2) sind zwar erfüllt, aber (G3) nicht, denn nicht \emph{jedes} Element hat ein inverses Element. $0$ hat eines, $1$ aber nicht, denn es gibt keine natürliche Zahl, die $n+1=1+n=0$ erfüllt. (Es gibt eine ganze Zahl, die das erfüllt, aber keine natürliche)

        \item Aus ähnlichen Gründen ist auch $(\IR,\cdot)$ keine Gruppe. Das Element $0\in\IR$ ist nicht invertierbar, denn keine reelle Zahl erfüllt $x\cdot 0 = 1$.
    \end{enumerate}
\end{example}


\begin{lemma}[Eindeutigkeit des neutralen Elements und der inversen Elemente] \label{gruppen:eindeutig_1_inv}
Sei $(G,*)$ eine Gruppe.
\begin{enumerate}[ref=\thetheorem.\alph*]
    \item \label{gruppen:eindeutig_1}
    Es gibt in $G$ genau ein neutrales Element.
    \item \label{gruppen:eindeutig_inv}
    Zu jedem $x \in G$ gibt es genau ein Inverses $x' \in G$. Dieses nennt man daher $x^{-1}$.
\end{enumerate}
\end{lemma}
\begin{proof}
    Übung.
\end{proof}

\begin{lemmadef}[Potenzschreibweise]
    Zusätzlich zur Schreibweise $g^{-1}$ für inverse Elemente führen wir allgemeiner Potenzen mit ganzzahligen Exponenten für alle Gruppenelemente ein.

    Ist $(G,\cdot)$ eine (multiplikativ notierte) Gruppe und $g\in G$, so definieren wir $g^k$, indem wir
    \[g^0 := 1 \quad g^{k+1}:=g^k\cdot g \quad\text{und}\quad g^{k-1} := g^k \cdot g^{-1}\]
    für alle $k\in\IZ$ festlegen. Mit dieser Bezeichnung gilt dann:
    \begin{enumerate}
        \item $\forall g\in G \forall n,m\in\IZ: g^{n+m} = g^n \cdot g^m$
        \item $\forall g\in G \forall n,m\in\IZ: g^{nm} = (g^n)^m$
    \end{enumerate}
    Falls $(G,\cdot)$ eine abelsche Gruppe ist, dann gilt außerdem
    \begin{enumerate}[resume]
        \item $\forall g_1,g_2\in G\forall n\in\IZ: (g_1 \cdot g_2)^n = g_1^n \cdot g_2^n$
    \end{enumerate}
\end{lemmadef}
\begin{proof}
    Übung.
\end{proof}

\begin{remark}
    In additiv geschriebenen Gruppen benutzt man normalerweise nicht die Potenzschreibweise $g^n$, sondern die Schreibweise $ng$. Dann schreiben sich die drei Potenzgesetze aber ebenso wiedererkennbar einfach als
    \[(n+m)g = ng+mg,\quad (nm)g = n(mg),\quad n(g_1+g_2)=ng_1+ng_2\]
\end{remark}
% !TeX root = main.tex
% !TeX spellcheck = de_DE

\subsection{Ringe}\label{sec:rings}
\begin{definition}
    Ein \udot{Ring} $R=(R,+,\cdot,0,1)$ besteht aus einer Menge $R$ zusammen mit Abbildungen ${+} : R \times R \to R$, $(x,y) \mapsto x+y$, genannt \enquote{Addition}, und $\cdot : R\times R \to R$, $(x,y) \mapsto x\cdot y = xy$, genannt \enquote{Multiplikation}, sowie zwei Elementen $0,1\in R$, welche die folgenden Bedingungen erfüllen:
    \begin{description}
        \item{(R1)} $(R,+,0)$ ist eine abelsche Gruppe. Das neutrale Element $0$ nennen wir \udot{Nullelement} des Rings.
        \item{(R1)} Assoziativität der Multiplikation:
        \[\forall x,y,z\in R: (xy)z = x(yz)\]
        \item{(R3)} Neutrales Element der Multiplikation:
        \[\exists 1_R\in R \forall x\in R: x\cdot 1=x=1\cdot x\]
        Dieses Element nennen wir \udot{Einselement} des Rings.
        \item{(R4)} Distributivgesetze:
        \[\forall x,y,z\in R: x(y+z)=xy+xz \wedge (x+y)z=xz+yz\]
    \end{description}
    Der Ring heißt \udot{kommutativ}, falls zusätzlich
    \begin{description}
        \item{(R5)} Kommutativität: $\forall x,y\in R: xy=yx$.
    \end{description}
    gilt.

    Der Kürze halber schreibt man oft nur $R$ statt $(R,+,\cdot,0,1)$, wenn aus dem Kontext klar ist, welche Addition und Multiplikation gemeint sind.
\end{definition}

\begin{example}
    \begin{enumerate}
        \item $\IZ$, $\IQ$ und $\IR$ sind Ringe bezüglich der üblichen Addition und
        Multiplikation.
        \item Der Nullring $R=\set{0}$ mit $0+0=0\cdot0=0$ ist ein Ring, mit $1_R=0_R=0$. In der Tat ist das der einzige Ring, in dem $0_R=1_R$ gilt (Übung).
        \item Ist $R$ ein Ring und $n\in\IN$, so ist die Menge $R^{n\times n}$ aller $n\times n$-Matrizen über $R$ ein Ring bezüglich Matrixaddition und -multiplikation (s. unten).
        \item Die Menge $R[X]$ aller Polynome mit Koeffizienten aus $R$ ist ein
        Ring zusammen mit der üblichen Addition:
        \[\sum_{i=0}^n a_i X^i + \sum_{i=0}^n b_i X^i := \sum_{i=0}^{n} (a_i+b_i)X^i\]
        \[(\sum_{i=0}^n a_i X^i)(\sum_{j=0}^m b_j X^j) := \sum_{k=0}^{n+m} (\sum_{\substack{i,j \\ i+j=k}} a_i b_j) X^k\]
        Ist $R$ kommutativ, dann ist auch $R[X]$ ein kommutativer Ring.
    \end{enumerate}
\end{example}

\begin{lemma}
    Sei $R$ ein Ring. Dann:
    \begin{enumerate}
        \item $\forall x \in R : 0\cdot x = x \cdot 0 = 0$.
        \item $\forall x,y \in R : x\cdot(-y)=(-x)\cdot y = -(xy)$.
        \item $(-1)^2 = 1$.
    \end{enumerate}
\end{lemma}

\begin{lemmadef}[Ideale und Quotienten]
    Es sei $R$ ein Ring. Eine Teilmenge $I\subseteq R$ heißt \udot{Ideal}, wenn
    \begin{description}
        \item[(I1)] $I$ ist eine Untergruppe von $(R,+)$, d.h.
        \[0\in I \wedge \forall x,y\in I: x+y\in I \wedge -x\in I\]
        \item[(I2)] $I$ ist unter Multiplikation mit beliebigen Elementen von $R$ abgeschlossen:
        \[\forall x\in I, y\in R: xy,yx\in I\]
    \end{description}

    Ist nun $I \unlhd R$ ein Ideal, dann gilt:
    \begin{enumerate}
        \item Die Relation
        \[x \equiv_I y :\iff x-y\in I\]
        ist eine Äquivalenzrelation auf $R$
        \item Der \udot{Quotientenring} $R/I$ ist definiert als die Menge der Äquivalenzklassen zusammen mit folgender Addition und Multiplikation
        \[[x] +_{R/I} [y] := [x +_R y],  \quad [x] \cdot_{R/I} [y] := [x \cdot_R y]\]
        Dies ist tatsächlich ein Ring.
    \end{enumerate}
\end{lemmadef}

\begin{example}
    In $R=\IZ$ sind die Ideale genau die Teilmengen $n\IZ:=\Set{nk | k \in\IZ}$ (Übung). Die Quotienten $\IZ/n\IZ$ nennt man auch \udot{Restklassenringe}.
\end{example}

\begin{theorem}[Chinesischer Restsatz]
    Es sei $R$ ein Ring und $I,J\subseteq R$ zwei Ideale. Die natürliche Abbildung
    \[ R\to R/I \times R/J, x\mapsto ([r]_{R/I}, [y]_{R/J})\]
    ist
    \begin{enumerate}
        \item injektiv genau dann, wenn $I\cap J = \set{0}$, und
        \item surjektiv genau dann, wenn $I+J=R$ oder äquivalent, wenn $\exists i\in I,j\in J: i+j=1$.
    \end{enumerate}
\end{theorem}

\subsubsection{Matrizen}

\begin{definition}
    Es sei $R$ ein Ring und $n,m\in\IN$. Eine \udot{$n\times m$-Matrix mit Einträgen aus $R$} ist eine rechteckige Anordnung von $nm$ Elementen von $R$ in $n$ Zeilen und $m$ Spalten. Im Deutschen Sprachgebrauch ist es üblich, runde Klammern um eine Matrix zu schreiben.

    Die Menge aller $n\times m$-Matrizen mit Einträgen aus $R$ bezeichnen wir mit $R^{n\times m}$.

    Ist $A$ eine solche Matrix, so bezeichnet man mit $A_{ij}$ den \udot{Eintrag an
    der Stelle $(i,j)$}, d.h. in der $i$ten Zeile und der $j$ten Spalte.

    Eine Matrix heißt \udot{quadratisch}, wenn die Anzahl der Zeilen und
    der Spalten gleich sind.
\end{definition}

\begin{example}
    Hier ist eine $(3\times2)$-Matrix mit Einträgen aus $\IR$:
    $\begin{pmatrix}
         1& 2\\4& -7\\0,5& \pi
    \end{pmatrix}$.
\end{example}

\begin{example}
    Eine $1\times 0$-Matrix: $()$.
\end{example}

\begin{example}
    Für $A=\begin{pmatrix}
               1& 3& 7& 4\\3& 1& 2& 9\\8& 0& 7& 3
    \end{pmatrix}$
    ist $A_{23}=2$.
\end{example}

\begin{definition}[Matrixaddition und -multiplikation]
    Für Matrizen $A,B \in R^{m\times n}$ wird die Summe
    $A+B \in R^{m\times n}$ definiert durch
    \[(A+B)_{ij} := A_{ij}+B_{ij}\]
    für alle $1\leq i\leq m, 1\leq j\leq n$.

    \medbreak
    Sind $A \in R^{m \times n}$ und $B \in R^{n \times p}$ Matrizen,
    so wird das Produkt $AB \in R^{m\times p}$ definiert durch
    \[(AB)_{ik} := A_{i1}B_{1k} + A_{i2}B_{2k} + A_{i3}B_{3k} + \cdots
    + A_{in} B_{nk} = \sum_{j=1}^n A_{ij} B_{jk} \]
    für alle $1\leq i\leq m, 1\leq j\leq n$.

    \medbreak
    Ist $A\in R^{m\times n}$ eine Matrix und $\lambda\in R$ ein Skalar, so ist $\lambda A\in R^{m\times n}$ definiert durch
    \[(\lambda A)_{ij} := \lambda A_{ij}\]

    \medbreak
    Die \udot{Nullmatrix} $0=0_{n\times m}\in R^{n\times m}$ ist definiert durch
    \[(0_{n\times m})_{ij} := 0\]

    \medbreak
    Die \udot{Einheitsmatrix} $1_{n\times n}$ ist die $n\times n$-Matrix, deren Eintrag an der Stelle $(i,j)$ genau
    \[\delta_{ij} = \begin{cases}
                        1 & i = j \\ 0 & \text{sonst}
    \end{cases}\]
    ist.
\end{definition}

\begin{example}
    $\begin{pmatrix}
         1& 2& 4\\2& 5& 7
    \end{pmatrix}
    + \begin{pmatrix}
          0& 3& 1\\1& 1& 1
    \end{pmatrix}
    = \begin{pmatrix}
          1& 5& 5\\3& 6& 8
    \end{pmatrix}$.
\end{example}
\begin{example}
    \[\left(\begin{array}{cc}
                1 & 0 \\
                3 & 1
    \end{array}\right)\cdot\left(\begin{array}{ccc}
                                     0 & 2 & 4 \\
                                     1 & 3 & 5
    \end{array}\right)
    = \begin{pmatrix}
          1\cdot0+0\cdot1 & 1 \cdot 2 + 0\cdot3 & 1\cdot4+0\cdot5   \\
          3\cdot0+1\cdot1 & 3\cdot2+1\cdot3     & 3\cdot4 + 1\cdot5
    \end{pmatrix}
    = \begin{pmatrix}
          0& 2& 4\\1& 9& 17
    \end{pmatrix} \]
\end{example}

\begin{remark}
    Beachten Sie: das Produkt $AB$ ist nur dann definiert, wenn $A$ die gleiche
    Anzahl von Spalten hat, wie $B$ Zeilen hat.
\end{remark}


\begin{lemma}[Ringeigenschaften]
    \label{matrizen:ring_eigenschaften}
    Matrizenringe $R^{n\times n}$ sind wirklich Ringe. Allgemeiner gilt selbst für rechteckige Matrizen:
    \begin{enumerate}
        \item $(R^{n\times m},+,0)$ ist eine abelsche Gruppe.
        \item Matrixmultiplikation ist assoziativ:
        \[\forall A\in R^{n\times m}, B\in R^{m\times p}, C\in R^{p\times q}: (AB)C = A(BC)\]
        \item Matrizenmultiplikation ist distributiv:
        \[\forall A\in R^{n\times m}, B,C\in R^{m\times p} : A(B+C) = AB+AC\]
        \[\forall A;B\in R^{n\times m}, C\in R^{m\times p} : (A+B)C = AC+BC\]
        \item Einheitsmatrizen sind Einselemente:
        \[\forall A \in R^{n \times m}: 1_{n\times n} \cdot A = A = A\cdot 1_{m\times m}\]
        \item Multiplikation mit Skalaren ist assoziativ:
        \[\forall A\in R^{n\times m}, B\in R^{m\times p}\forall \lambda\in R: (\lambda A)B = \lambda (AB)\]
        sowie, wenn $R$ kommutativ ist
        \[\forall A\in R^{n\times m}, B\in R^{m\times p}\forall \lambda\in R: \lambda (AB)=A(\lambda B)\]
    \end{enumerate}
\end{lemma}

\begin{proof}
    Übung.
\end{proof}

\begin{example}
    Schon $R^{2\times 2}$ und alle größeren Matrizenringe sind nicht kommutativ, selbst wenn $R$ noch kommutativ war, denn
    \[
        \begin{pmatrix}
            1& 0\\0& 0
        \end{pmatrix}
        \begin{pmatrix}
            0& 1\\0& 0
        \end{pmatrix}
        = \begin{pmatrix}
              0& 1\\0& 0
        \end{pmatrix}
        \quad \text{aber} \quad
        \begin{pmatrix}
            0& 1\\0& 0
        \end{pmatrix}
        \begin{pmatrix}
            1& 0\\0& 0
        \end{pmatrix}
        = \begin{pmatrix}
              0& 0\\0& 0
        \end{pmatrix} \, .
    \]
\end{example}

\begin{sheet}

    \begin{problem}[title={Gruppen}]
        Beweise alle Behauptungen in Abschnitt \label{groups}
    \end{problem}

    \begin{problem}[title={Ringe}]
        Beweise alle Behauptungen in Abschnitt \label{rings}
    \end{problem}

    \begin{problem}[title={Quotienten von Polynomringen I}]
        Es sei $R$ ein kommutativer Ring. Betrachte den Polynomring $R[X]$ in einer Unbekannten und ein normiertes Polynom $p(X) = X^n + p_{n-1} X^{n-1} + \ldots + p_1 x+p_0 \in R[X]$ vom Grad $n$. Betrachte das von $p$ erzeugte \udot{Hauptideal} $(p):=pR[X]:=\Set{q\cdot p | q\in R[X]}$.


        \begin{subproblem}
            Jede Restklasse von $R[X]/(p)$ enthält genau ein Polynom vom Grad $\leq n-1$.

            Dieses kann auch effizient berechnet werden: Ist ein beliebiges Polynom
            \[f = f_0 + f_1 X + f_2 X ^2 + \ldots f_m X^m\]
            gegeben, so kann das Polynom $\tilde{f}$ vom Grad $<n$ mit $\tilde{f} \equiv f\mod p$ in $O(nm)$ ausgerechnet werden.
        \end{subproblem}

        \begin{subproblem}
            Etwas abstrakter: Die additive Gruppe von $R[X]/(p)$ ist via
            \[(a_0,a_1,\ldots,a_{n-1}) \mapsto [a_0+a_1 X+\ldots +a_{n-1}X^{n-1}]\]
            zu $R^n$ isomorph.
        \end{subproblem}

        \begin{subproblem}
            Wenn $n=1$, also $p(X) = X-p_0$ ist, dann ist das in der Tat ein Isomorphismus von Ringen. Die Umkehrabbildung ist die Auswertungsabbildung $R[X]/(X-p_0) \to R, [f] \mapsto f(p_0)$.
        \end{subproblem}
    \end{problem}

    \begin{problem}[title={Ideale vs. Teilbarkeit}]
        Historisch sind Ideale als \enquote{ideale Zahlen} eingeführt worden, um bestimmte Eigenschaften der Teilbarkeitsrelation besser zu verstehen. Dies beruht auf folgenden zu zeigenden Eigenschaften.

        Es sei $R$ ein kommutativer Ring, $p,q\in R$ zwei beliebige Elemente. Zeige:
        \begin{subproblem}
            $(p) \subseteq (q) \iff q | p$
        \end{subproblem}

        \begin{subproblem}
            Gibt es ein $r\in R$ mit $(p) + (q) = (r)$, dann ist $r$ ein größter gemeinsamer Teiler von $p$ und $q$.
        \end{subproblem}
        \begin{subproblem}
            Gibt es ein $r\in R$ mit $(p) \cap (q) = (r)$, dann ist $r$ ein kleinstes gemeinsames Vielfaches von $p$ und $q$.
        \end{subproblem}
    \end{problem}

    \begin{problem}
        Zeige: $\IZ$ ist ein \udot{Hauptidealring}, d.h.\ jedes Ideal $I\unlhd\IZ$ ist von der Form $I=(k)$ für ein $k\in\IZ$.
    \end{problem}
    \begin{problem}
        Es sei $\IK$ ein Körper. Zeige, dass $\IK[X]$ auch ein Hauptidealring ist.
    \end{problem}
\end{sheet}

\printbibliography

\end{document}
