% !TeX root = main.tex
% !TeX spellcheck = de_DE

\subsection{Ringe}\label{ringe}
\begin{definition}
    Ein \udot{Ring} $R=(R,+,\cdot,0,1)$ besteht aus einer Menge $R$ zusammen mit Abbildungen ${+} : R \times R \to R$, $(x,y) \mapsto x+y$, genannt \enquote{Addition}, und $\cdot : R\times R \to R$, $(x,y) \mapsto x\cdot y = xy$, genannt \enquote{Multiplikation}, sowie zwei Elementen $0,1\in R$, welche die folgenden Bedingungen erfüllen:
    \begin{description}
        \item{(R1)} $(R,+,0)$ ist eine abelsche Gruppe. Das neutrale Element $0$ nennen wir \udot{Nullelement} des Rings.
        \item{(R1)} Assoziativität der Multiplikation:
        \[\forall x,y,z\in R: (xy)z = x(yz)\]
        \item{(R3)} Neutrales Element der Multiplikation:
        \[\exists 1_R\in R \forall x\in R: x\cdot 1=x=1\cdot x\]
        Dieses Element nennen wir \udot{Einselement} des Rings.
        \item{(R4)} Distributivgesetze:
        \[\forall x,y,z\in R: x(y+z)=xy+xz \wedge (x+y)z=xz+yz\]
    \end{description}
    Der Ring heißt \udot{kommutativ}, falls zusätzlich
    \begin{description}
        \item{(R5)} Kommutativität: $\forall x,y\in R: xy=yx$.
    \end{description}
    gilt.

    Der Kürze halber schreibt man oft nur $R$ statt $(R,+,\cdot,0,1)$, wenn aus dem Kontext klar ist, welche Addition und Multiplikation gemeint sind.
\end{definition}

\begin{example}
    \begin{enumerate}
        \item $\IZ$, $\IQ$ und $\IR$ sind Ringe bezüglich der üblichen Addition und
        Multiplikation.
        \item Der Nullring $R=\set{0}$ mit $0+0=0\cdot0=0$ ist ein Ring, mit $1_R=0_R=0$. In der Tat ist das der einzige Ring, in dem $0_R=1_R$ gilt (Übung).
        \item Ist $R$ ein Ring und $n\in\IN$, so ist die Menge $R^{n\times n}$ aller $n\times n$-Matrizen über $R$ ein Ring bezüglich Matrixaddition und -multiplikation (s. unten).
        \item Die Menge $R[X]$ aller Polynome mit Koeffizienten aus $R$ ist ein
        Ring zusammen mit der üblichen Addition:
        \[\sum_{i=0}^n a_i X^i + \sum_{i=0}^n b_i X^i := \sum_{i=0}^{n} (a_i+b_i)X^i\]
        \[(\sum_{i=0}^n a_i X^i)(\sum_{j=0}^m b_j X^j) := \sum_{k=0}^{n+m} (\sum_{\substack{i,j \\ i+j=k}} a_i b_j) X^k\]
        Ist $R$ kommutativ, dann ist auch $R[X]$ ein kommutativer Ring.
    \end{enumerate}
\end{example}

\begin{lemma}
    Sei $R$ ein Ring. Dann:
    \begin{enumerate}
        \item $\forall x \in R : 0\cdot x = x \cdot 0 = 0$.
        \item $\forall x,y \in R : x\cdot(-y)=(-x)\cdot y = -(xy)$.
        \item $(-1)^2 = 1$.
    \end{enumerate}
\end{lemma}

\begin{lemmadef}[Ideale und Quotienten]
    Es sei $R$ ein Ring. Eine Teilmenge $I\subseteq R$ heißt \udot{Ideal}, wenn
    \begin{description}
        \item[(I1)] $I$ ist eine Untergruppe von $(R,+)$, d.h.
        \[0\in I \wedge \forall x,y\in I: x+y\in I \wedge -x\in I\]
        \item[(I2)] $I$ ist unter Multiplikation mit beliebigen Elementen von $R$ abgeschlossen:
        \[\forall x\in I, y\in R: xy,yx\in I\]
    \end{description}

    Ist nun $I \unlhd R$ ein Ideal, dann gilt:
    \begin{enumerate}
        \item Die Relation
        \[x \equiv_I y :\iff x-y\in I\]
        ist eine Äquivalenzrelation auf $R$
        \item Der \udot{Quotientenring} $R/I$ ist definiert als die Menge der Äquivalenzklassen zusammen mit folgender Addition und Multiplikation
        \[[x] +_{R/I} [y] := [x +_R y],  \quad [x] \cdot_{R/I} [y] := [x \cdot_R y]\]
        Dies ist tatsächlich ein Ring.
    \end{enumerate}
\end{lemmadef}

\begin{example}
    In $R=\IZ$ sind die Ideale genau die Teilmengen $n\IZ:=\Set{nk | k \in\IZ}$ (Übung). Die Quotienten $\IZ/n\IZ$ nennt man auch \udot{Restklassenringe}.
\end{example}

\begin{theorem}[Chinesischer Restsatz]
    Es sei $R$ ein Ring und $I,J\subseteq R$ zwei Ideale. Die natürliche Abbildung
    \[ R\to R/I \times R/J, x\mapsto ([r]_{R/I}, [y]_{R/J})\]
    ist
    \begin{enumerate}
        \item injektiv genau dann, wenn $I\cap J = \set{0}$, und
        \item surjektiv genau dann, wenn $I+J=R$ oder äquivalent, wenn $\exists i\in I,j\in J: i+j=1$.
    \end{enumerate}
\end{theorem}

\subsubsection{Matrizen}

\begin{definition}
    Es sei $R$ ein Ring und $n,m\in\IN$. Eine \udot{$n\times m$-Matrix mit Einträgen aus $R$} ist eine rechteckige Anordnung von $nm$ Elementen von $R$ in $n$ Zeilen und $m$ Spalten. Im Deutschen Sprachgebrauch ist es üblich, runde Klammern um eine Matrix zu schreiben.

    Die Menge aller $n\times m$-Matrizen mit Einträgen aus $R$ bezeichnen wir mit $R^{n\times m}$.

    Ist $A$ eine solche Matrix, so bezeichnet man mit $A_{ij}$ den \udot{Eintrag an
    der Stelle $(i,j)$}, d.h. in der $i$ten Zeile und der $j$ten Spalte.

    Eine Matrix heißt \udot{quadratisch}, wenn die Anzahl der Zeilen und
    der Spalten gleich sind.
\end{definition}

\begin{example}
    Hier ist eine $(3\times2)$-Matrix mit Einträgen aus $\IR$:
    $\begin{pmatrix}
         1& 2\\4& -7\\0,5& \pi
    \end{pmatrix}$.
\end{example}

\begin{example}
    Eine $1\times 0$-Matrix: $()$.
\end{example}

\begin{example}
    Für $A=\begin{pmatrix}
               1& 3& 7& 4\\3& 1& 2& 9\\8& 0& 7& 3
    \end{pmatrix}$
    ist $A_{23}=2$.
\end{example}

\begin{definition}[Matrixaddition und -multiplikation]
    Für Matrizen $A,B \in R^{m\times n}$ wird die Summe
    $A+B \in R^{m\times n}$ definiert durch
    \[(A+B)_{ij} := A_{ij}+B_{ij}\]
    für alle $1\leq i\leq m, 1\leq j\leq n$.

    \medbreak
    Sind $A \in R^{m \times n}$ und $B \in R^{n \times p}$ Matrizen,
    so wird das Produkt $AB \in R^{m\times p}$ definiert durch
    \[(AB)_{ik} := A_{i1}B_{1k} + A_{i2}B_{2k} + A_{i3}B_{3k} + \cdots
    + A_{in} B_{nk} = \sum_{j=1}^n A_{ij} B_{jk} \]
    für alle $1\leq i\leq m, 1\leq j\leq n$.

    \medbreak
    Ist $A\in R^{m\times n}$ eine Matrix und $\lambda\in R$ ein Skalar, so ist $\lambda A\in R^{m\times n}$ definiert durch
    \[(\lambda A)_{ij} := \lambda A_{ij}\]

    \medbreak
    Die \udot{Nullmatrix} $0=0_{n\times m}\in R^{n\times m}$ ist definiert durch
    \[(0_{n\times m})_{ij} := 0\]

    \medbreak
    Die \udot{Einheitsmatrix} $1_{n\times n}$ ist die $n\times n$-Matrix, deren Eintrag an der Stelle $(i,j)$ genau
    \[\delta_{ij} = \begin{cases}
                        1 & i = j \\ 0 & \text{sonst}
    \end{cases}\]
    ist.
\end{definition}

\begin{example}
    $\begin{pmatrix}
         1& 2& 4\\2& 5& 7
    \end{pmatrix}
    + \begin{pmatrix}
          0& 3& 1\\1& 1& 1
    \end{pmatrix}
    = \begin{pmatrix}
          1& 5& 5\\3& 6& 8
    \end{pmatrix}$.
\end{example}
\begin{example}
    \[\left(\begin{array}{cc}
                1 & 0 \\
                3 & 1
    \end{array}\right)\cdot\left(\begin{array}{ccc}
                                     0 & 2 & 4 \\
                                     1 & 3 & 5
    \end{array}\right)
    = \begin{pmatrix}
          1\cdot0+0\cdot1 & 1 \cdot 2 + 0\cdot3 & 1\cdot4+0\cdot5   \\
          3\cdot0+1\cdot1 & 3\cdot2+1\cdot3     & 3\cdot4 + 1\cdot5
    \end{pmatrix}
    = \begin{pmatrix}
          0& 2& 4\\1& 9& 17
    \end{pmatrix} \]
\end{example}

\begin{remark}
    Beachten Sie: das Produkt $AB$ ist nur dann definiert, wenn $A$ die gleiche
    Anzahl von Spalten hat, wie $B$ Zeilen hat.
\end{remark}


\begin{lemma}[Ringeigenschaften]
    \label{matrizen:ring_eigenschaften}
    Matrizenringe $R^{n\times n}$ sind wirklich Ringe. Allgemeiner gilt selbst für rechteckige Matrizen:
    \begin{enumerate}
        \item $(R^{n\times m},+,0)$ ist eine abelsche Gruppe.
        \item Matrixmultiplikation ist assoziativ:
        \[\forall A\in R^{n\times m}, B\in R^{m\times p}, C\in R^{p\times q}: (AB)C = A(BC)\]
        \item Matrizenmultiplikation ist distributiv:
        \[\forall A\in R^{n\times m}, B,C\in R^{m\times p} : A(B+C) = AB+AC\]
        \[\forall A;B\in R^{n\times m}, C\in R^{m\times p} : (A+B)C = AC+BC\]
        \item Einheitsmatrizen sind Einselemente:
        \[\forall A \in R^{n \times m}: 1_{n\times n} \cdot A = A = A\cdot 1_{m\times m}\]
        \item Multiplikation mit Skalaren ist assoziativ:
        \[\forall A\in R^{n\times m}, B\in R^{m\times p}\forall \lambda\in R: (\lambda A)B = \lambda (AB)\]
        sowie, wenn $R$ kommutativ ist
        \[\forall A\in R^{n\times m}, B\in R^{m\times p}\forall \lambda\in R: \lambda (AB)=A(\lambda B)\]
    \end{enumerate}
\end{lemma}

\begin{proof}
    Übung.
\end{proof}

\begin{example}
    Schon $R^{2\times 2}$ und alle größeren Matrizenringe sind nicht kommutativ, selbst wenn $R$ noch kommutativ war, denn
    \[
        \begin{pmatrix}
            1& 0\\0& 0
        \end{pmatrix}
        \begin{pmatrix}
            0& 1\\0& 0
        \end{pmatrix}
        = \begin{pmatrix}
              0& 1\\0& 0
        \end{pmatrix}
        \quad \text{aber} \quad
        \begin{pmatrix}
            0& 1\\0& 0
        \end{pmatrix}
        \begin{pmatrix}
            1& 0\\0& 0
        \end{pmatrix}
        = \begin{pmatrix}
              0& 0\\0& 0
        \end{pmatrix} \, .
    \]
\end{example}
