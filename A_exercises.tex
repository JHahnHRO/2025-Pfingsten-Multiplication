\begin{sheet}

    \begin{problem}[title={Gruppen}]
        Beweise alle Behauptungen in Abschnitt~\ref{sec:groups}
    \end{problem}

    \begin{problem}[title={Ringe}]
        Beweise alle Behauptungen in Abschnitt~\ref{sec:rings}
    \end{problem}

    \begin{problem}[title={Quotienten von Polynomringen I}]
        Es sei $R$ ein kommutativer Ring. Betrachte den Polynomring $R[X]$ in einer Unbekannten und ein normiertes Polynom $p(X) = X^n + p_{n-1} X^{n-1} + \ldots + p_1 x+p_0 \in R[X]$ vom Grad $n$. Betrachte das von $p$ erzeugte \udot{Hauptideal} $(p):=pR[X]:=\Set{q\cdot p | q\in R[X]}$.


        \begin{subproblem}
            Jede Restklasse von $R[X]/(p)$ enthält genau ein Polynom vom Grad $\leq n-1$.

            Dieses kann auch effizient berechnet werden: Ist ein beliebiges Polynom
            \[f = f_0 + f_1 X + f_2 X ^2 + \ldots f_m X^m\]
            gegeben, so kann das Polynom $\tilde{f}$ vom Grad $<n$ mit $\tilde{f} \equiv f\mod p$ in $O(nm)$ ausgerechnet werden.
        \end{subproblem}

        \begin{subproblem}
            Etwas abstrakter: Die additive Gruppe von $R[X]/(p)$ ist via
            \[(a_0,a_1,\ldots,a_{n-1}) \mapsto [a_0+a_1 X+\ldots +a_{n-1}X^{n-1}]\]
            zu $R^n$ isomorph.
        \end{subproblem}

        \begin{subproblem}
            Wenn $n=1$, also $p(X) = X-p_0$ ist, dann ist das in der Tat ein Isomorphismus von Ringen. Die Umkehrabbildung ist die Auswertungsabbildung $R[X]/(X-p_0) \to R, [f] \mapsto f(p_0)$.
        \end{subproblem}
    \end{problem}

    \begin{problem}[title={Ideale vs. Teilbarkeit}]
        Historisch sind Ideale als \enquote{ideale Zahlen} eingeführt worden, um bestimmte Eigenschaften der Teilbarkeitsrelation besser zu verstehen. Dies beruht auf folgenden zu zeigenden Eigenschaften.

        Es sei $R$ ein kommutativer Ring, $p,q\in R$ zwei beliebige Elemente. Zeige:
        \begin{subproblem}
            $(p) \subseteq (q) \iff q \mid p$
        \end{subproblem}

        \begin{subproblem}
            Gibt es ein $r\in R$ mit $(p) + (q) = (r)$, dann ist $r$ ein größter gemeinsamer Teiler von $p$ und $q$.
        \end{subproblem}
        \begin{subproblem}
            Gibt es ein $r\in R$ mit $(p) \cap (q) = (r)$, dann ist $r$ ein kleinstes gemeinsames Vielfaches von $p$ und $q$.
        \end{subproblem}
    \end{problem}

    \begin{problem}
        Zeige: $\IZ$ ist ein \udot{Hauptidealring}, d.h.\ jedes Ideal $I\unlhd\IZ$ ist von der Form $I=(k)$ für ein $k\in\IZ$.
    \end{problem}
    \begin{problem}
        Es sei $\IK$ ein Körper. Zeige, dass $\IK[X]$ auch ein Hauptidealring ist.
    \end{problem}

    \begin{problem}
        Aus dem chinesischen Restsatz folgt, dass $\IK[X]/(X^{2k}-z^2)$ zu $\IK[X]/(X^k-z) \times \IK[X]/(X^k+z)$ isomorph ist. Finde eine explizite Abbildungsvorschrift für die Umkehrabbildung.
    \end{problem}
\end{sheet}