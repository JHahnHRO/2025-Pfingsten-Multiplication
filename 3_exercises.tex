% !TeX root = main.tex
% !TeX spellcheck = de_DE

\begin{sheet}

\begin{problem}[title={Vandermonde-Determinante}, difficulty={mittel}]
Beweise, dass die Vandermonde\footnote{Alexandre-Théophile Vandermonde, 1735--1796, französischer Mathematiker, Musiker und Chemiker}-Matrix
\[V:=\begin{pmatrix}
s_0^0 & s_0^1 & \cdots & s_0^d \\
s_1^0 & s_1^1 & \cdots & s_1^d \\
\vdots & \vdots & \ddots & \vdots \\
s_d^0 & s_d^1 & \cdots & s_d^d
\end{pmatrix}\]
die Determinante
\[\det(V) = \prod_{0\leq i<j\leq d+1} (s_j-s_i)\]
hat. Insbesondere ist $V$ genau dann invertierbar, wenn die $s_i$ paarweise verschieden sind.
\end{problem}

\begin{problem}[difficulty={leicht}]
Bestimme die Determinante der analogen Matrix für die Stützstellen $s_0,s_1,\ldots,s_{d-1},\infty$.
\end{problem}

\begin{problem}
Verallgemeinere den Toom-Cook-$k$-Algorithmus zum Toom-Cook-$(k_1,k_2)$-Algorithmus, der $A$ in $k_1$ und $B$ in $k_2$ kleinere Stücke zerlegt.
\end{problem}

\begin{remark}
Der Toom-Cook-$(2,3)$-Algorithmus wird manchmal auch als Toom-Cook-$2.5$ bezeichnet.
\end{remark}

\end{sheet}