% !TeX root = main.tex
% !TeX spellcheck = de_DE

\begin{sheet}

\begin{problem}[title={Mastertheorem}]\label{ex:mastertheorem}
Beweise das \emph{Mastertheorem für divide-and-conquer-Algorithmen}:

Gegeben sei eine Rekursion der Form
\[T(n) = a T\left(\frac{n}{b}\right) + f(n)\]
für Zahlen $a,b>1$ und eine nichtnegative Funktion $f$. Definiere $c_0 := \frac{\log(a)}{\log(b)}$. Zeige:
\begin{subproblem}[difficulty={leicht}]
Wenn $f\in O(n^c)$ mit $c<c_0$, dann ist $T\in\Theta(n^{c_0})$. Speziell für $f(n) := d n^c$ ist sogar $T(n) = k_0 n^{c_0} + k n^c$ mit geeigneten Konstanten $k_0,k$.
\end{subproblem}
\begin{subproblem}[difficulty={leicht}]
Wenn $f\in O(n^{c_0}\log(n)^k)$ mit $k\geq 0$, dann ist $T\in O(n^{c_0} \log(n)^{k+1})$.
\end{subproblem}
\begin{subproblem}[difficulty={mittel}]
Wenn $f\in o(n^c)$ mit $c>c_0$, dann ist $T\in o(n^c)$. Wenn zusätzlich $a f(\frac{n}{b})\leq k f(n)$ für eine Konstante $0\leq k<1$ und alle hinreichend großen $n$ gilt, dann ist $T\in\Theta(f)$.
\end{subproblem}
\end{problem}

\end{sheet}