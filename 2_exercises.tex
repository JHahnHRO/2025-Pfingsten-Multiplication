% !TeX root = main.tex
% !TeX spellcheck = de_DE

\begin{sheet}

    \begin{problem}[title={Mastertheorem}]
        \label{ex:mastertheorem}
        Beweise das \emph{Mastertheorem für divide-and-conquer-Algorithmen}:

        Gegeben sei eine Rekursion der Form
        \[T(n) = a T\left(\frac{n}{b}\right) + f(n)\]
        für Zahlen $a,b>1$ und eine nichtnegative Funktion $f$. Definiere $c_0 := \frac{\log(a)}{\log(b)}$. Zeige:
        \begin{subproblem}[difficulty={leicht}]
            Wenn $f\in O(n^c)$ mit $c<c_0$, dann ist $T\in\Theta(n^{c_0})$. Speziell für $f(n) := d n^c$ ist sogar $T(n) = k_0 n^{c_0} + k n^c$ mit geeigneten Konstanten $k_0,k$.
        \end{subproblem}
        \begin{subproblem}[difficulty={leicht}]
            Wenn $f\in O(n^{c_0}\log(n)^k)$ mit $k\geq 0$, dann ist $T\in O(n^{c_0} \log(n)^{k+1})$.
        \end{subproblem}
        \begin{subproblem}[difficulty={mittel}]
            Wenn $f\in o(n^c)$ mit $c>c_0$, dann ist $T\in o(n^c)$. Wenn zusätzlich $a f(\frac{n}{b})\leq k f(n)$ für eine Konstante $0\leq k<1$ und alle hinreichend großen $n$ gilt, dann ist $T\in\Theta(f)$.
        \end{subproblem}
    \end{problem}

    \begin{problem}
        Wende das Mastertheorem auf das Beispiel von Sortieralgorithmen an: Viele Sortieralgorithmen wie Heapsort, Mergesort, Quicksort, \ldots arbeiten nach folgendem Prinzip:
        \begin{enumerate}
            \item Teile die zu sortierende Liste in zwei Hälften
            \item sortiere die beiden Hälften mittels rekursiver Anwendung des gleichen Algorithmus
            \item Iteriere einmal über beide (nun sortierte) Teillisten um eine sortierte Gesamtliste zu erstellen.
        \end{enumerate}
        Zeige, dass alle diese Sortieralgorithmen Zeitkomplexität $O(n\log(n))$ haben.
    \end{problem}

    \begin{problem}[title={Strassens Matrixmultiplikationsalgorithmus}]
        Analog zu Karatsubas Trick für Multiplikation ganzer Zahlen gibt es Strassens Trick für Multiplikation von $2\times 2$-Matrizen $A,B$: Wenn wir das Produkt mit $X$ bezeichnen, also
        \[ \begin{pmatrix}
               X_{11} & X_{12} \\ X_{21} & X_{22}
        \end{pmatrix} :=
        \begin{pmatrix}
            A_{11} & A_{12} \\ A_{21} & A_{22}
        \end{pmatrix} \cdot
        \begin{pmatrix}
            B_{11} & B_{12} \\ B_{21} & B_{22}
        \end{pmatrix} = \begin{pmatrix}
                            A_{11} B_{11}+ A_{12} B_{21}  & A_{11} B_{12} + A_{12} B_{22} \\
                            A_{21} B_{11} + A_{22} B_{21} & A_{21} B_{12} +A_{22} B_{22}
        \end{pmatrix}\]
        dann gilt
        \[\begin{pmatrix}
              X_{11} & X_{12} \\ X_{21} & X_{22}
        \end{pmatrix} =
        \begin{pmatrix}
            M_1 + M_4 - M_5 + M_7 & M_3 + M_5             \\
            M_2 + M_4             & M_1 - M_2 + M_3 + M_6
        \end{pmatrix}\]
        wobei
        \begin{align*}
            M_1 &:= (A_{11} + A_{22}) \cdot (B_{11} + B_{22}) \\
            M_2 &:= (A_{21} + A_{22}) \cdot B_{11} \\
            M_3 &:= A_{11} \cdot (B_{12} - B_{22}) \\
            M_4 &:= A_{22} \cdot (B_{21} - B_{11}) \\
            M_5 &:= (A_{11} + A_{12}) \cdot B_{22} \\
            M_6 &:= (A_{21} - A_{11}) \cdot (B_{11} + B_{12}) \\
            M_7 &:= (A_{12} - A_{22}) \cdot (B_{21} + B_{22}) \\
        \end{align*}
        Daraus ergibt sich ein divide-and-conquer-Algorithmus, naheligenderweise \enquote{Strassen-Algorithmus} genannt. Bestimme die Laufzeit- und Speicherkomplexität dieses Algorithmus.
    \end{problem}

    \begin{remark}
        Um $n\times n$-Matrizen zu multiplizieren genügen also $7$ Multiplikationen von $\frac{n}{2}\times\frac{n}{2}$-Matrizen. Indem man Strassens Trick zweimal anwendet, kann man also auch $7^2=49$ Multiplikationen von $\frac{n}{4}\times\frac{n}{4}$-Matrizen verwenden.

        Google hat nur wenige Tage vor unserem Kurs bekannt gegeben, dass deren AI \enquote{AlphaEvolve} ein gewisses Optimierungsproblem lösen konnte und dabei zu dem Ergebnis gekommen ist, dass man mit nur $48$ Multiplikationen auskommt, was einen leicht besseren Algorithmus als Strassens ergibt; der Exponent ist ca. 0,53\% kleiner. \cite{googleAlphaEvolve}
    \end{remark}

\end{sheet}