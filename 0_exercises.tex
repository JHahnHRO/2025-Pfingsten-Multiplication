\begin{sheet}

    \begin{problem}[title={Gruppen}]
        Beweise alle Behauptungen in Abschnitt 0.1
    \end{problem}

    \begin{problem}[title={Ringe}]
        Beweise alle Behauptungen in Abschnitt 0.2
    \end{problem}

    \begin{problem}[title={Quotienten von Polynomringen I}]
        Es sei $R$ ein kommutativer Ring. Betrachte den Polynomring $R[X]$ in einer Unbekannten und ein normiertes Polynom $p(X) = x^n + p_{n-1} X^{n-1} + \ldots + p_1 x+p_0 \in R[X]$ vom Grad $n$. Betrachte das von $p$ erzeugte \udot{Hauptideal} $(p):=pR[X]:=\Set{q\cdot p | q\in R[X]}$.


        \begin{subproblem}
            Jede Restklasse von $R[X]/(p)$ enthält genau ein Polynom vom Grad $\leq n-1$.

            Dieses kann auch effizient berechnet werden: Ist ein beliebiges Polynom
            \[f = f_0 + f_{1X + f_2 X ^2 + \ldots f_m X^m}\]
            gegeben, so kann das Polynom $\tilde{f}$ vom Grad $<n$ mit $\tilde{f} \equiv f$ in $O(m)$ Additionen ausgerechnet werden.
        \end{subproblem}

        \begin{subproblem}
            Etwas abstrakter: Die additive Gruppe von $R[X]/(p)$ ist via
            \[(a_0,a_1,\ldots,a_{n-1}) \mapsto [a_0+a_1 X+\ldots +a_{n-1}X^{n-1}]\]
            zu $R^n$ isomorph.
        \end{subproblem}

        \begin{subproblem}
            Wenn $n=1$, also $p(X) = x+p_0$ ist, dann ist das in der Tat ein Isomorphismus von Ringen. Die Umkehrabbildung ist die Auswertungsabbildung $R[X]/(x+p_0) \to R, [f] \mapsto f(-p_0)$.
        \end{subproblem}
    \end{problem}

    \begin{problem}[title={Ideale vs. Teilbarkeit}]
        Historisch sind Ideale als \enquote{ideale Zahlen} eingeführt worden, um bestimmte Eigenschaften der Teilbarkeitsrelation besser zu verstehen. Dies beruht auf folgenden zu zeigenden Eigenschaften.

        Es sei $R$ ein kommutativer Ring, $p,q\in R$ zwei beliebige Elemente. Zeige:
        \begin{subproblem}
            $(p) \subseteq (q) \iff q | p$
        \end{subproblem}

        Ist $R=\IZ$ oder $R=\IK[X]$ für einen Körper $\IK$, so gilt außerdem:
        \begin{subproblem}
            $(p) + (q) = (ggT(p,q))$
        \end{subproblem}
        \begin{subproblem}
            $(p) \cap (q) = (kgV(p,q))$
        \end{subproblem}
    \end{problem}
\end{sheet}