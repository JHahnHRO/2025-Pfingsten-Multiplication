% !TeX root = main.tex
% !TeX spellcheck = de_DE

\begin{sheet}
    \begin{problem}
        Betrachte den folgenden Algorithmus für Multiplikation von $l\times m$- und $m\times n$-Matrizen.
        \begin{lstlisting}
Matrix multiply(Matrix a, Matrix b){
    int l = nrOfRows(a);
    int m = nrOfColumns(a);
    if(m != nrOfRows(b)){
        throw new ArithmeticException("Matrix multiplication undefined");
    }
    int n = nrOfColumns(b);

    var C = new Matrix(l,n);

    for(int i=0; i<l; i++){
        for(int j=0; j<n; j++){
            for(int k=0; k<m; k++){
                C[i,j] += A[i,k] * B[k,j];
            }
        }
    }
    return C;
}
        \end{lstlisting}

        Welche Laufzeit- und Speicher-Komplexität hat dieser Algorithmus um zwei $n\times n$-Matrizen miteinander zu multiplizieren?
    \end{problem}

    \begin{problem}[title={Ein paar übliche Komplexitätsklassen}]
        \begin{subproblem}
            Es seien $r,s$ reelle Zahlen. Zeige
            \[r<s \implies O(n^r) \subsetneq O(n^s)\]
            Zeige insbesondere, dass dies eine echte Inklusion ist, d.h.\ dass es eine Funktion $g\in O(n^s) \setminus O(n^r)$ gibt.
        \end{subproblem}
        \begin{subproblem}
            Zeige $\log(n),\log(\log(n)), \log(\log(\log(n))), \ldots \in O(n^\varepsilon)$ für jedes $\varepsilon>0$.
        \end{subproblem}
        \begin{subproblem}
            Zeige $n^r \in O(c^n)$ für jedes $r>0$ und $c>1$.
        \end{subproblem}
    \end{problem}

    \begin{problem}
        Es seien $f,g,F,G:\IN\to\IR_{\geq 0}$ Funktionen mit $f\in O(F), g\in O(G)$.
        \begin{subproblem}
            Zeige, dass $f+g\in O(F+G)$.
        \end{subproblem}
        \begin{subproblem}
            Zeige, dass $f\cdot g\in O(F\cdot G)$.
        \end{subproblem}
        \begin{subproblem}
            Folgere, dass $\log(n)^r, \log(\log(n))^r, \log(\log(\log(n)))^r, \ldots$ alle in $O(n^\varepsilon)$ sind für jedes $\varepsilon>0$.

            Folgere allgemeiner, dass Funktionen der Form $n^s\cdot \log(n)^{r_1} \cdot \log(\log(n))^{r_2} \cdot \log(\log\log(n))^{r_2}\cdots$ alle in $O(n^{s+\varepsilon})$ sind für jedes $\varepsilon>0$.
        \end{subproblem}
    \end{problem}

\end{sheet}